\chapter{
  VBS Measurement in ZVjj Final State
 }\label{ch_vbs}

As discussed in Section~\ref{ch_intro:vbs} this analysis targets
\gls{VBS} of ZV with two jets in final state. The goal of the analysis
is to reduce contribution from background processes as much as possible
without losing much of signal, and measure signal strength and significance.

Since Z decays leptonically and V is decaying hadronically,
the phase-space of this analysis can be either
\( l^+ l^- jjjj \) or \( l^+ l^- J jj\), where \( l \) are leptons (electrons
or muons),
\( j \) are narrow jets and \( J \) is a boosted (wider) jet.
The phase-space is divided into two broad regions signal and controlled,
signal region is constructed based on theory such that it is mostly signal process
and control region is basically orthogonal to the signal region
where we expect contributions mostly from background processes.
The main purpose of control region
in this analysis is to
provide normalization factors the dominant background.

The analysis is performed ``blind'' to avoid intrinsic bias
i.e.\ until the analysis procedure is finalized, the collision data is only used
in control regions. Once the analysis technique is optimized using \gls{MC}
samples and validated against collision data in control regions. Once the
analysis technique is satisfactory and approved by \gls{CMS} Physics Group
then the results are ``un-blinded'' i.e.\ measurements are done
using collision data in signal region.

\section{
  Dataset and Simulation
 }

As discussed in Section~\ref{ch_cms:L1T} only events those pass Level-1
Trigger and \gls{HLT} paths are saved for further processing, \gls{MC}
simulation also have these identical step during event generation to mimic
Level-1 Trigger and \gls{HLT} paths.

\gls{CMS} collaboration processes the datasets centrally and provides various
tiers of datasets such as ``RECO'' datasets, which contains reconstructed
objects and no skimming. Average size of an event saved at ``RECO'' tier is
480 kB per event and on average an analysis will process
3 billion events, which makes this tier not practical in terms of computer processing
time and storage if each analysis starts from ``RECO''. \gls{CMS} centrally
processes these datasets further and removes certain objects
or features to reduce the average event size but still covering
majority of the analysis to make use of the reduced datasets.

This analysis uses \NanoAOD{} tier with version ``v7'' of datasets
which has average event size of 2--3 kB.

\subsection{
  Data
}

The collision data events used in this analysis are all certified
by \gls{CMS} \gls{DQM} and \gls{DC} group. The primary trigger object in
\gls{HLT} paths are leptons \( p_{T} \) threshold, and since in our final state
we are looking for Z boson decaying into two leptons, we require single and
double lepton trigger for our analysis. Depending on the detector
conditions and \gls{LHC} storage capacity we have slightly different threshold
in triggers across different years. The Table~\ref{tab:hlt-paths} contains the list of
\gls{HLT} paths used in this analysis.

\begin{table}[!ht]
  \centering
  \caption{Trigger paths used to select events in CMS collision data}
  \resizebox{\textwidth}{!}{%
    \begin{tabular}{clll}%
      \toprule
      Dataset & Year                  & HLT Path                                          & \( p_T \) Threshold \\
      \midrule
      \multirow{4}{*}{Single Muon}
              & \multirow{2}{*}{2016} & \url{HLT_IsoMu24}                                 & 24 \GeV{}           \\
              &                       & \url{HLT_IsoTkMu24}                               & 24 \GeV{}           \\
      \cmidrule(lr){2-4}
              & \multirow{1}{*}{2017} & \url{HLT_IsoMu27}                                 & 27 \GeV{}           \\
      \cmidrule(lr){2-4}
              & \multirow{1}{*}{2018} & \url{HLT_IsoMu24}                                 & 24 \GeV{}           \\
      \midrule
      \multirow{4}{*}{Single Electron}
              & \multirow{2}{*}{2016} & \url{HLT_Ele27_WPTight_Gsf}                       & 27 \GeV{}           \\
              &                       & \url{HLT_Ele25_eta2p1_WPTight_Gsf}                & 25 \GeV{}           \\
      \cmidrule(lr){2-4}
              & \multirow{1}{*}{2017} & \url{HLT_Ele35_WPTight_Gsf}                       & 35 \GeV{}           \\
      \cmidrule(lr){2-4}
              & \multirow{1}{*}{2018} & \url{HLT_Ele32_WPTight_Gsf}                       & 32 \GeV{}           \\
      \midrule
      \multirow{6}{*}{Double Muon}
              & \multirow{3}{*}{2016} & \url{HLT_Mu17_TrkIsoVVL_Mu8_TrkIsoVVL_DZ}         & 17, 8 \GeV{}        \\
              &                       & \url{HLT_Mu17_TrkIsoVVL_TkMu8_TrkIsoVVL_DZ}       & 17, 8 \GeV{}        \\
              &                       & \url{HLT_TkMu17_TrkIsoVVL_TkMu8_TrkIsoVVL_DZ}     & 17, 8 \GeV{}        \\
      \cmidrule(lr){2-4}
              & \multirow{2}{*}{2017} & \url{HLT_Mu17_TrkIsoVVL_Mu8_TrkIsoVVL_DZ_Mass8}   & 17, 8 \GeV{}        \\
              &                       & \url{HLT_Mu17_TrkIsoVVL_Mu8_TrkIsoVVL_DZ_Mass3p8} & 17, 8 \GeV{}        \\
      \cmidrule(lr){2-4}
              & \multirow{1}{*}{2018} & \url{HLT_Mu17_TrkIsoVVL_Mu8_TrkIsoVVL_DZ_Mass3p8} & 17, 8  \GeV{}       \\
      \midrule
      \multirow{4}{*}{Double Electron}
              & \multirow{2}{*}{2016} & \url{HLT_Ele23_Ele12_CaloIdL_TrackIdL_IsoVL_DZ}   & 23, 12 \GeV{}       \\
              &                       & \url{HLT_Ele23_Ele12_CaloIdL_TrackIdL_IsoVL}      & 23, 12 \GeV{}       \\
      \cmidrule(lr){2-4}
              & \multirow{1}{*}{2017} & \url{HLT_Ele23_Ele12_CaloIdL_TrackIdL_IsoVL}      & 23, 12 \GeV{}       \\
      \cmidrule(lr){2-4}
              & \multirow{1}{*}{2018} & \url{HLT_Ele23_Ele12_CaloIdL_TrackIdL_IsoVL}      & 23, 12 \GeV{}       \\
      \bottomrule
    \end{tabular}}\label{tab:hlt-paths}
\end{table}

\subsection{
  MC Simulations
}

The \gls{EW} \gls{VBS} process which is our signal is
generated with \MADGRAPH{}5+\PYTHIA{}8 at \gls{LO} with \( \alpha_{EW}^{6} \) order
i.e.
all vertices in tree level Feynman diagram are \gls{EW} vertices.
The \gls{QCD} induced \gls{VBS} background process which is very
similar to our signal is generated with same configuration
but with \( \alpha_{EW}^{4} \alpha_{QCD}^{2} \) order i.e.
two of the six vertices are \gls{QCD}. The dominant background to analysis
\gls{DY} + Jets and, top-quark based processes consisting of
single top-quark (t and s-channel),
single top-quark in association with W boson (tW) and top-quark pair (t\=t) production.
The \gls{DY} + Jets are generated at \gls{LO} with \MADGRAPH{}5+\PYTHIA{}8
in bins of HT i.e.
scalar sum of all the jets \( p_{T} \) in the event, to have more statistics
for higher HT bins. Top-quark background process are generated at \gls{NNLO},
s-channel is generated with \MGvATNLO+\PYTHIA{}8 and others
t-channel, tW, t\=t are generated with \POWHEG{}+\PYTHIA{}8.
The complete list of Table~\ref{tab:mc-list-1},~\ref{tab:mc-list-2},~\ref{tab:mc-list-3} contains
the list of Signal and Background MC samples used for modeling in this analysis.

\begin{sidewaystable}[!ht]
  \centering
  \caption{List of MC samples for Signal and Background modeling}
  \resizebox{\textwidth}{!}{%
    \begin{tabular}{llll}%
      \toprule
      Process & Year       & Dataset Name                                                                      & Cross Section (pb) \\
      \midrule
      \midrule
      \multirow{6}{1.1in}{
        VBS\_EWK (Signal)}
              & 2016       & \url{WminusTo2JZTo2LJJ_EWK_LO_SM_MJJ100PTJ10_TuneCUETP8M1_13TeV-madgraph-pythia8} & 0.02982            \\
              & 2016       & \url{WplusTo2JZTo2LJJ_EWK_LO_SM_MJJ100PTJ10_TuneCUETP8M1_13TeV-madgraph-pythia8}  & 0.05401            \\
              & 2016       & \url{ZTo2LZTo2JJJ_EWK_LO_SM_MJJ100PTJ10_TuneCUETP8M1_13TeV-madgraph-pythia8}      & 0.01589            \\
      \cmidrule(lr){2-4}
              & 2017, 2018 & \url{WminusTo2JZTo2LJJ_EWK_LO_SM_MJJ100PTJ10_TuneCP5_13TeV-madgraph-pythia8}      & 0.02982            \\
              & 2017, 2018 & \url{WplusTo2JZTo2LJJ_EWK_LO_SM_MJJ100PTJ10_TuneCP5_13TeV-madgraph-pythia8}       & 0.05401            \\
              & 2017, 2018 & \url{ZTo2LZTo2JJJ_EWK_LO_SM_MJJ100PTJ10_TuneCP5_13TeV-madgraph-pythia8}           & 0.01589            \\
      \midrule
      \midrule
      \multirow{6}{1.1in}{
        VBS\_QCD (Background)}
              & 2016       & \url{WminusTo2JZTo2LJJ_QCD_LO_SM_MJJ100PTJ10_TuneCUETP8M1_13TeV-madgraph-pythia8} & 0.3488             \\
              & 2016       & \url{WplusTo2JZTo2LJJ_QCD_LO_SM_MJJ100PTJ10_TuneCUETP8M1_13TeV-madgraph-pythia8}  & 0.575              \\
              & 2016       & \url{ZTo2LZTo2JJJ_QCD_LO_SM_MJJ100PTJ10_TuneCUETP8M1_13TeV-madgraph-pythia8}      & 0.3449             \\
      \cmidrule(lr){2-4}
              & 2017, 2018 & \url{WminusTo2JZTo2LJJ_QCD_LO_SM_MJJ100PTJ10_TuneCP5_13TeV-madgraph-pythia8}      & 0.3488             \\
              & 2017, 2018 & \url{WplusTo2JZTo2LJJ_QCD_LO_SM_MJJ100PTJ10_TuneCP5_13TeV-madgraph-pythia8}       & 0.575              \\
              & 2017, 2018 & \url{ZTo2LZTo2JJJ_QCD_LO_SM_MJJ100PTJ10_TuneCP5_13TeV-madgraph-pythia8}           & 0.3449             \\
      \midrule
      \midrule
      \multirow{16}{1.1in}{
        DY + Jets LO (Background)}
              & 2016       & \url{DYJetsToLL_M-50_HT-70to100_TuneCUETP8M1_13TeV-madgraphMLM-pythia8}           & 169.9              \\
              & 2016       & \url{DYJetsToLL_M-50_HT-100to200_TuneCUETP8M1_13TeV-madgraphMLM-pythia8}          & 147.4              \\
              & 2016       & \url{DYJetsToLL_M-50_HT-200to400_TuneCUETP8M1_13TeV-madgraphMLM-pythia8}          & 41.04              \\
              & 2016       & \url{DYJetsToLL_M-50_HT-400to600_TuneCUETP8M1_13TeV-madgraphMLM-pythia8}          & 5.674              \\
              & 2016       & \url{DYJetsToLL_M-50_HT-600to800_TuneCUETP8M1_13TeV-madgraphMLM-pythia8}          & 1.358              \\
              & 2016       & \url{DYJetsToLL_M-50_HT-800to1200_TuneCUETP8M1_13TeV-madgraphMLM-pythia8}         & 0.6229             \\
              & 2016       & \url{DYJetsToLL_M-50_HT-1200to2500_TuneCUETP8M1_13TeV-madgraphMLM-pythia8}        & 0.1512             \\
              & 2016       & \url{DYJetsToLL_M-50_HT-2500toInf_TuneCUETP8M1_13TeV-madgraphMLM-pythia8}         & 0.003659           \\
      \cmidrule(lr){2-4}
              & 2017       & \url{DYJetsToLL_M-50_HT-70to100_TuneCP5_13TeV-madgraphMLM-pythia8}                & 167.33             \\
              & 2017       & \url{DYJetsToLL_M-50_HT-100to200_TuneCP5_13TeV-madgraphMLM-pythia8}               & 161.1              \\
              & 2017       & \url{DYJetsToLL_M-50_HT-200to400_TuneCP5_13TeV-madgraphMLM-pythia8}               & 48.66              \\
              & 2017       & \url{DYJetsToLL_M-50_HT-400to600_TuneCP5_13TeV-madgraphMLM-pythia8}               & 6.968              \\
              & 2017       & \url{DYJetsToLL_M-50_HT-600to800_TuneCP5_13TeV-madgraphMLM-pythia8}               & 1.743              \\
              & 2017       & \url{DYJetsToLL_M-50_HT-800to1200_TuneCP5_13TeV-madgraphMLM-pythia8}              & 0.8052             \\
              & 2017       & \url{DYJetsToLL_M-50_HT-1200to2500_TuneCP5_13TeV-madgraphMLM-pythia8}             & 0.1933             \\
              & 2017       & \url{DYJetsToLL_M-50_HT-2500toInf_TuneCP5_13TeV-madgraphMLM-pythia8}              & 0.003468           \\
      \bottomrule
    \end{tabular}%
  }\label{tab:mc-list-1}
\end{sidewaystable}

\begin{sidewaystable}[!ht]
  \centering
  \caption{List of MC samples for Signal and Background modeling}
  \resizebox{\textwidth}{!}{%
    \begin{tabular}{llll}%
      \toprule
      Process & Year & Dataset Name                                                                         & Cross Section (pb) \\
      \midrule
      \midrule
      \multirow{8}{1.1in}{
        DY + Jets LO (Background)}
              & 2018 & \url{DYJetsToLL_M-50_HT-70to100_TuneCP5_PSweights_13TeV-madgraphMLM-pythia8}         & 167.33             \\
              & 2018 & \url{DYJetsToLL_M-50_HT-100to200_TuneCP5_PSweights_13TeV-madgraphMLM-pythia8}        & 161.1              \\
              & 2018 & \url{DYJetsToLL_M-50_HT-200to400_TuneCP5_PSweights_13TeV-madgraphMLM-pythia8}        & 48.66              \\
              & 2018 & \url{DYJetsToLL_M-50_HT-400to600_TuneCP5_PSweights_13TeV-madgraphMLM-pythia8}        & 6.968              \\
              & 2018 & \url{DYJetsToLL_M-50_HT-600to800_TuneCP5_PSweights_13TeV-madgraphMLM-pythia8}        & 1.743              \\
              & 2018 & \url{DYJetsToLL_M-50_HT-800to1200_TuneCP5_PSweights_13TeV-madgraphMLM-pythia8}       & 0.8052             \\
              & 2018 & \url{DYJetsToLL_M-50_HT-1200to2500_TuneCP5_PSweights_13TeV-madgraphMLM-pythia8}      & 0.1933             \\
              & 2018 & \url{DYJetsToLL_M-50_HT-2500toInf_TuneCP5_PSweights_13TeV-madgraphMLM-pythia8}       & 0.003468           \\
      \midrule
      \midrule
      \multirow{21}{1.1in}{
        Top (Background)}
              & 2016 & \url{ST_tW_antitop_5f_NoFullyHadronicDecays_13TeV_PSweights-powheg-pythia8}          & 38.06              \\
              & 2016 & \url{ST_tW_antitop_5f_inclusiveDecays_TuneCP5_PSweights_13TeV-powheg-pythia8}        & 34.97              \\
              & 2016 & \url{ST_tW_top_5f_NoFullyHadronicDecays_13TeV_PSweights-powheg-pythia8}              & 38.09              \\
              & 2016 & \url{ST_tW_top_5f_inclusiveDecays_TuneCP5_PSweights_13TeV-powheg-pythia8}            & 34.91              \\
              & 2016 & \url{ST_t-channel_antitop_4f_InclusiveDecays_TuneCP5_PSweights_13TeV-powheg-pythia8} & 67.91              \\
              & 2016 & \url{ST_t-channel_top_4f_InclusiveDecays_TuneCP5_PSweights_13TeV-powheg-pythia8}     & 113.3              \\
              & 2016 & \url{ST_s-channel_4f_leptonDecays_13TeV_PSweights-amcatnlo-pythia8}                  & 3.365              \\
              & 2016 & \url{ST_s-channel_4f_hadronicDecays_TuneCP5_PSweights_13TeV-amcatnlo-pythia8}        & 11.24              \\
              & 2016 & \url{ST_s-channel_4f_InclusiveDecays_13TeV-amcatnlo-pythia8}                         & 10.12              \\
              & 2016 & \url{TTToHadronic_TuneCP5_PSweights_13TeV-powheg-pythia8}                            & 377.96             \\
              & 2016 & \url{TTToSemiLeptonic_TuneCP5_PSweights_13TeV-powheg-pythia8}                        & 365.34             \\
              & 2016 & \url{TTTo2L2Nu_TuneCP5_PSweights_13TeV-powheg-pythia8}                               & 88.29              \\
      \cmidrule(lr){2-4}
              & 2017 & \url{TTToSemiLeptonic_TuneCP5_13TeV-powheg-pythia8}                                  & 365.34             \\
              & 2017 & \url{TTTo2L2Nu_TuneCP5_13TeV-powheg-pythia8}                                         & 86.99              \\
              & 2017 & \url{TTToHadronic_TuneCP5_13TeV-powheg-pythia8}                                      & 377.96             \\
              & 2017 & \url{ST_s-channel_antitop_leptonDecays_13TeV-PSweights_powheg-pythia}                & 1.33               \\
              & 2017 & \url{ST_s-channel_top_leptonDecays_13TeV-PSweights_powheg-pythia}                    & 2.13               \\
              & 2017 & \url{ST_t-channel_antitop_5f_TuneCP5_PSweights_13TeV-powheg-pythia8}                 & 27.19              \\
              & 2017 & \url{ST_t-channel_top_5f_TuneCP5_13TeV-powheg-pythia8}                               & 45.7               \\
              & 2017 & \url{ST_tW_antitop_5f_inclusiveDecays_TuneCP5_13TeV-powheg-pythia8}                  & 12.04              \\
              & 2017 & \url{ST_tW_top_5f_inclusiveDecays_TuneCP5_13TeV-powheg-pythia8}                      & 12.04              \\
      \bottomrule
    \end{tabular}%
  }\label{tab:mc-list-2}
\end{sidewaystable}

\begin{sidewaystable}[!ht]
  \centering
  \caption{List of MC samples for Signal and Background modeling}
  \resizebox{\textwidth}{!}{%
    \begin{tabular}{llll}%
      \toprule
      Process & Year & Dataset Name                                                           & Cross Section (pb) \\
      \midrule
      \midrule
      \multirow{9}{1.1in}{
        Top (Background)}
              & 2018 & \url{TTToSemiLeptonic_TuneCP5_13TeV-powheg-pythia8}                    & 365.34             \\
              & 2018 & \url{TTTo2L2Nu_TuneCP5_13TeV-powheg-pythia8}                           & 86.99              \\
              & 2018 & \url{TTToHadronic_TuneCP5_13TeV-powheg-pythia8}                        & 377.96             \\
              & 2018 & \url{ST_s-channel_antitop_leptonDecays_13TeV-PSweights_powheg-pythia}  & 1.33               \\
              & 2018 & \url{ST_s-channel_top_leptonDecays_13TeV-PSweights_powheg-pythia}      & 2.13               \\
              & 2018 & \url{ST_t-channel_antitop_5f_TuneCP5_13TeV-powheg-pythia8}             & 27.19              \\
              & 2018 & \url{ST_t-channel_top_5f_TuneCP5_13TeV-powheg-pythia8}                 & 45.7               \\
              & 2018 & \url{ST_tW_DS_antitop_5f_inclusiveDecays_TuneCP5_13TeV-powheg-pythia8} & 12.04              \\
              & 2018 & \url{ST_tW_DS_top_5f_inclusiveDecays_TuneCP5_13TeV-powheg-pythia8}     & 12.04              \\
      \bottomrule
    \end{tabular}%
  }\label{tab:mc-list-3}
\end{sidewaystable}

\clearpage
\section{
  Event Selection
 }

In first stage of selection events are selected if an events has minimum number
of objects for analysis categories i.e.
at least two leptons of same flavor (\( p_T > 10 \GeV{} \)), and either
four narrow jets (AK4) (Resolved ZV category) or two narrow jets (AK4)
plus one wider jet (AK8) (Boosted ZV category).

After initial skimming step, leptons are selected with \gls{OSSF} for Z candidate
with following selections for muon and electron channels:
\begin{itemize}
  \item \textbf{Muons}: Muons with \( p_T < 10 \GeV{} \), \( |\eta| > 2.4 \),
        failing loose ID, and \gls{PF} relative isolation in cone \( R = 0.4 \) (``pfRelIso04'') more than
        \( 0.40 \) are vetoed~\cite{cms-muon-id}.
        Then exactly two tight muons with opposite charge are selected with \( p_T > 20 \GeV{} \), passing
        tight ID, ``pfRelIso04'' less than 0.15 and impact parameters \( d_{xy} < 0.01, d_z < 0.1 \),
        for all the years.
  \item \textbf{Electrons}: Electrons with \( p_T < 10 \GeV{} \), \( |\eta| > 2.5 \),
        and failing loose cut based~\cite{cms-egamma-id}.
        Tight selection of electron is different 2016 and 2017--2018,
        with \( p_T > 20 \GeV \) same for all years, for 2016 year electrons with
        passing ``mvaSpring16GP\_WP90'' (MVA based ID, 90\% efficiency) and, ``pfRelIso03\_all'' less than \( 0.0571 \) for
        \( |\eta| > 1.479 \) and less than \( 0.0588 \) otherwise,
        for 2017 and 2018 years, electrons passing ``mvaFall17V2Iso\_WP90'' (MVA based ID + Isolation, 90\% efficiency) and ``pfRelIso03\_all''
        less than \( 0.06 \) is required.
  \item \textbf{\textit{Z} Candidate}: After lepton selection, events with
        \textit{Z} candidates with mass in range \( [76, 106] \GeV{} \) are selected.
\end{itemize}

\subsection{
  VBS Tagged Jets and \textit{V} Jet Candidate
}
After selecting leptons and constructing \textit{Z} candidate, first we look
for \textit{V} as a ``FatJet'' (merged jet, AK8) candidate with \( p_T \) of 200 \GeV{},
within \( |\eta| < 2.4 \) and in softdrop mass range of [40, 150] \GeV{}.
Then all FatJets isolation with selected leptons are checked with \( \Delta R = 0.8 \)
and then N-subjetiness \( \tau_{21} < 0.45 \) cut is applied.
Finally a FatJet with mass nearest to the average mass of \textit{W} and \textit{Z}
is selected.

\begin{itemize}
  \item \textbf{Boosted}: Now after above described selection,
        if we have a FatJet as a candidate for \textit{V} then the event is
        categorized as ``boosted''.
  \item \textbf{VBS Jets}: Now whether the event is a boosted event or not,
        narrow jets (AK4) are considered to be tagged as VBS jets,
        this is done by first cleaning jets by applying tight ``JetID'',
        \( p_T > 30 \), \( |\eta| < 2.4 \),
        and checking isolation against selected leptons with \(\Delta R = 0.4\),
        selected FatJet (if any) with \( \Delta R = 0.8 \)
        and other jets with \(\Delta R = 0.4\).
        Then all the combinations of jets are checked as a pair,
        to find the pair with highest invariant mass with minimum of 500 \GeV{},
        and the jets forming that pair are tagged as VBS jets.
        If no such pair is found, the event is not selected.
  \item \textbf{Resolved}: If there is no boosted event but there
        is a of VBS tagged jets, then from rest of jet collection a pair
        is searched with invariant mass in range of [40, 150] \GeV{} and
        nearest to the average mass of \textit{W} and \textit{Z}.
        Jets from selected pair are then labelled as \textit{V} jet candidates
        and event is categorized as ``resolved''.
\end{itemize}

\section{
  Control and Signal Region
 }

To model our background as accurate as possible,
the selected events are divided into two regions based usually
a observed property of the selected objects.

For this analysis, the mass window of \textit{V} is selected
to define the region boundaries and two regions are defined as:

\begin{itemize}
  \item \textbf{Signal Region}: \( 65 < M_{V} < 105 \) \GeV{},
        it's an 20 \GeV{} window around the mass of \textit{V}. This is a
        narrow mass window, and we expect cleaner signal in this region.
  \item \textbf{Control Region}: \( 40 < M_{V} < 65 \) and \( 105 < M_{V} < 150 \) \GeV{},
        in these mass sidebands around signal region are rich
        in background. This region is dominant in \gls{DY} plus jets
        events and will be used to model and normalize this background process.
\end{itemize}

\subsection{
  MC Scale Factors
}

Whenever we apply some cut on object properties on data and \gls{MC},
the efficiency of the cut can be different for the data than it's for
\gls{MC}. To account for this difference in efficiency
of data and \gls{MC}, scale factors are calculated and applied
to \gls{MC}. The scale factors are usually function of
\( p_T \) and \( \eta \) of the object being used for selection.
These scale factors are then applied as event weight i.e.
they are multiplied to the cross-section weight factor applied to the \gls{MC}
events.
Following are the scale factors used in this analysis:

\begin{itemize}
  \item \textbf{Pile-Up Reweighting}: During run2 data taking at \gls{CMS}
        the average number of pileup in an event can varied from 5 to 100,
        and to have same shape for this variable in \gls{MC}, reweighting
        scale derived from the difference is applied to \gls{MC}.
  \item \textbf{Trigger Efficiency}: Since the \gls{HLT} trigger paths are defined
        by the \( p_T \) threshold of the object in the trigger,
        they can have different efficiency in data and \gls{MC}.
  \item \textbf{Lepton Identification/Isolation}:
        Scale factors to account for tight selection
        of electrons and muons are
        derived from ``tag and probe'' method and applied to the \gls{MC}.
  \item \textbf{Tight Pile-Up Jet Identification}:
        Since to reduce jets coming from pileup vertex, tight Pile-Up Identification
        is applied to the \gls{VBS} tagged jets, a scale factor
        is calculated for both \gls{VBS} jets and their product
        is applied to \gls{MC}.
  \item \textbf{L1 Prefire Correction}: L1 Trigger prefiring
        scale factors recommended by \gls{CMS} for 2016 and 2017 year datasets~\cite{cms-l1t}.
\end{itemize}

\subsection{
  DY+Jets Normalization
}

To normalize and model our main background \gls{MC} DY+Jets,
we extract scale factors from bins of variables
in control region and apply it to both control and signal region.
For boosted category of all years 2016, 2017, 2018
and resolved in 2016 1D bins of \( p_T \) of \textit{Z} were used, and
for resoled in 2016 and 2017 2D bins of \( p_T \) of \textit{Z} and VBS trailing
jet were used. The Table lists the normalization
factors extracted in 1D bin of \( p_T \) of \textit{Z} and
Figure~\ref{fig:norm-factors-2d} lists factors extracted
in 2D bin of \( p_T \) of \textit{Z} and VBS trailing jet.

\begin{table}
  \caption{Normalization factors from 1D bin of \( p_T \) of \textit{Z}}
  \centering
  \begin{tabular}{lcccccccc}
    \toprule
                           & \multicolumn{2}{c}{Resolved} & \multicolumn{6}{c}{Boosted}                                                                                                   \\
    \cmidrule(lr){2-9}
    \textit{Z} \(p_T\) bin & \multicolumn{2}{c}{2016}     & \multicolumn{2}{c}{2016}    & \multicolumn{2}{c}{2017} & \multicolumn{2}{c}{2018}                                             \\
    \midrule
                           & \( e \)                      & \( \mu \)                   & \( e \)                  & \( \mu \)                & \( e \) & \( \mu \) & \( e \) & \( \mu \) \\
    \cmidrule(lr){2-9}\relax
    [0, 80]                & 1.11                         & 1.19                        & 1.03                     & 1.09                     & 1.02    & 0.92      & 0.86    & 1.01      \\\relax
    [80, 160]              & 1.07                         & 1.12                        & 1.03                     & 1.00                     & 0.84    & 0.83      & 0.86    & 0.81      \\\relax
    [160, 240]             & 1.14                         & 1.13                        & 0.88                     & 1.04                     & 0.80    & 0.82      & 0.75    & 0.68      \\\relax
    [240, 320]             & 0.93                         & 1.14                        & 1.15                     & 0.83                     & 0.93    & 0.99      & 0.73    & 0.76      \\\relax
    [320, 400]             & 1.07                         & 1.08                        & 0.82                     & 0.73                     & 0.85    & 1.23      & 0.70    & 0.76      \\\relax
    [400, 480]             & 1.02                         & 1.06                        & 0.70                     & 0.62                     & 0.84    & 0.91      & 0.67    & 0.67      \\\relax
    [480, inf]             & 0.84                         & 0.81                        & 0.56                     & 0.40                     & 0.76    & 0.45      & 0.64    & 0.68      \\
    \bottomrule
  \end{tabular}\label{tab:norm-factors}
\end{table}

\begin{figure}
  \centering
  \includegraphics[width=0.45\textwidth]{analysis_plots/dyjet_norm_2d/cr_vjets_e_sf_from_2017_zjj.pdf}
  \includegraphics[width=0.45\textwidth]{analysis_plots/dyjet_norm_2d/cr_vjets_mu_sf_from_2017_zjj.pdf}\\
  \includegraphics[width=0.45\textwidth]{analysis_plots/dyjet_norm_2d/cr_vjets_e_sf_from_2018_zjj.pdf}
  \includegraphics[width=0.45\textwidth]{analysis_plots/dyjet_norm_2d/cr_vjets_mu_sf_from_2018_zjj.pdf}
  \caption[Normalization factors from 1D bin of \( p_T \) of \textit{Z}
    and VBS trailing jet]%
  {Normalization factors from 1D bin of \( p_T \) of \textit{Z}
    and VBS trailing jet. Top to Bottom: 2017 and 2018 resolved.
    Left to Right: electron and muon channel.}%
  \label{fig:norm-factors-2d}
\end{figure}


\clearpage
\section{Kinematics Distributions}

To understand how the kinematics of various final state objects and
derived candidate behave, they are studied in the control
region by stacking up all the background histograms and
overlaying them with the data as points.
Signal \gls{MC} is also overlayed on the histograms, since it's small,
for visualization purposes it's scaled with a factor shown in
the legends of the plots.

The kinematics plots also include a ratio plot of data and total background
\gls{MC}, and uncertainties in histograms of \gls{MC} are shown as shaded region
which includes total statistical bin error and
theoretical renormalization error on DY+Jets and VBS\_QCD \gls{MC} samples.

Section~\ref{ch_vbs:boosted-plots} and~\ref{ch_vbs:resolved-plots}
shows kinematics plots in control region for boosted and
resolved ZV category respectively.

\subsection{
  Boosted ZV DY+Jets Control Region
}\label{ch_vbs:boosted-plots}

\begin{figure}[!ht]
  \centering
  \includegraphics[width=0.30\textwidth]{analysis_plots/2016_zv/cr_vjets_e/lep1_pt.pdf}
  \includegraphics[width=0.30\textwidth]{analysis_plots/2017_zv/cr_vjets_e/lep1_pt.pdf}
  \includegraphics[width=0.30\textwidth]{analysis_plots/2018_zv/cr_vjets_e/lep1_pt.pdf} \\
  \includegraphics[width=0.30\textwidth]{analysis_plots/2016_zv/cr_vjets_e/lep1_eta.pdf}
  \includegraphics[width=0.30\textwidth]{analysis_plots/2017_zv/cr_vjets_e/lep1_eta.pdf}
  \includegraphics[width=0.30\textwidth]{analysis_plots/2018_zv/cr_vjets_e/lep1_eta.pdf} \\
  \includegraphics[width=0.30\textwidth]{analysis_plots/2016_zv/cr_vjets_e/lep1_phi.pdf}
  \includegraphics[width=0.30\textwidth]{analysis_plots/2017_zv/cr_vjets_e/lep1_phi.pdf}
  \includegraphics[width=0.30\textwidth]{analysis_plots/2018_zv/cr_vjets_e/lep1_phi.pdf} \\
  \caption[DY+Jets Control Region: Leading electron kinematics in Boosted ZV Channel]%
  {DY+Jets Control Region: Leading electron kinematics in Boosted ZV Channel. From Left to Right: 2016,
    2017, and 2018. From Top to Bottom: \( p_T \), \( \eta \), \( \phi \).}%
  \label{fig:zv-cr-vjets-e-lep1-pt-eta-phi}
\end{figure}

\begin{figure}[!ht]
  \centering
  \includegraphics[width=0.30\textwidth]{analysis_plots/2016_zv/cr_vjets_e/lep2_pt.pdf}
  \includegraphics[width=0.30\textwidth]{analysis_plots/2017_zv/cr_vjets_e/lep2_pt.pdf}
  \includegraphics[width=0.30\textwidth]{analysis_plots/2018_zv/cr_vjets_e/lep2_pt.pdf} \\
  \includegraphics[width=0.30\textwidth]{analysis_plots/2016_zv/cr_vjets_e/lep2_eta.pdf}
  \includegraphics[width=0.30\textwidth]{analysis_plots/2017_zv/cr_vjets_e/lep2_eta.pdf}
  \includegraphics[width=0.30\textwidth]{analysis_plots/2018_zv/cr_vjets_e/lep2_eta.pdf} \\
  \includegraphics[width=0.30\textwidth]{analysis_plots/2016_zv/cr_vjets_e/lep2_phi.pdf}
  \includegraphics[width=0.30\textwidth]{analysis_plots/2017_zv/cr_vjets_e/lep2_phi.pdf}
  \includegraphics[width=0.30\textwidth]{analysis_plots/2018_zv/cr_vjets_e/lep2_phi.pdf} \\
  \caption[DY+Jets Control Region: Trailing electron kinematics in Boosted ZV Channel]%
  {DY+Jets Control Region: Trailing electron kinematics in Boosted ZV Channel. From Left to Right: 2016,
    2017, and 2018. From Top to Bottom: \( p_T \), \( \eta \), \( \phi \).}%
  \label{fig:zv-cr-vjets-e-lep2-pt-eta-phi}
\end{figure}

\begin{figure}[!ht]
  \centering
  \includegraphics[width=0.30\textwidth]{analysis_plots/2016_zv/cr_vjets_m/lep1_pt.pdf}
  \includegraphics[width=0.30\textwidth]{analysis_plots/2017_zv/cr_vjets_m/lep1_pt.pdf}
  \includegraphics[width=0.30\textwidth]{analysis_plots/2018_zv/cr_vjets_m/lep1_pt.pdf} \\
  \includegraphics[width=0.30\textwidth]{analysis_plots/2016_zv/cr_vjets_m/lep1_eta.pdf}
  \includegraphics[width=0.30\textwidth]{analysis_plots/2017_zv/cr_vjets_m/lep1_eta.pdf}
  \includegraphics[width=0.30\textwidth]{analysis_plots/2018_zv/cr_vjets_m/lep1_eta.pdf} \\
  \includegraphics[width=0.30\textwidth]{analysis_plots/2016_zv/cr_vjets_m/lep1_phi.pdf}
  \includegraphics[width=0.30\textwidth]{analysis_plots/2017_zv/cr_vjets_m/lep1_phi.pdf}
  \includegraphics[width=0.30\textwidth]{analysis_plots/2018_zv/cr_vjets_m/lep1_phi.pdf} \\
  \caption[DY+Jets Control Region: Leading muon kinematics in Boosted ZV Channel]%
  {DY+Jets Control Region: Leading muon kinematics in Boosted ZV Channel. From Left to Right: 2016,
    2017, and 2018. From Top to Bottom: \( p_T \), \( \eta \), \( \phi \).}%
  \label{fig:zv-cr-vjets-m-lep1-pt-eta-phi}
\end{figure}

\begin{figure}[!ht]
  \centering
  \includegraphics[width=0.30\textwidth]{analysis_plots/2016_zv/cr_vjets_m/lep2_pt.pdf}
  \includegraphics[width=0.30\textwidth]{analysis_plots/2017_zv/cr_vjets_m/lep2_pt.pdf}
  \includegraphics[width=0.30\textwidth]{analysis_plots/2018_zv/cr_vjets_m/lep2_pt.pdf} \\
  \includegraphics[width=0.30\textwidth]{analysis_plots/2016_zv/cr_vjets_m/lep2_eta.pdf}
  \includegraphics[width=0.30\textwidth]{analysis_plots/2017_zv/cr_vjets_m/lep2_eta.pdf}
  \includegraphics[width=0.30\textwidth]{analysis_plots/2018_zv/cr_vjets_m/lep2_eta.pdf} \\
  \includegraphics[width=0.30\textwidth]{analysis_plots/2016_zv/cr_vjets_m/lep2_phi.pdf}
  \includegraphics[width=0.30\textwidth]{analysis_plots/2017_zv/cr_vjets_m/lep2_phi.pdf}
  \includegraphics[width=0.30\textwidth]{analysis_plots/2018_zv/cr_vjets_m/lep2_phi.pdf} \\
  \caption[DY+Jets Control Region: Trailing muon kinematics in Boosted ZV Channel]%
  {DY+Jets Control Region: Trailing muon kinematics in Boosted ZV Channel. From Left to Right: 2016,
    2017, and 2018. From Top to Bottom: \( p_T \), \( \eta \), \( \phi \).}%
  \label{fig:zv-cr-vjets-m-lep2-pt-eta-phi}
\end{figure}

\begin{figure}[!ht]
  \centering
  \includegraphics[width=0.30\textwidth]{analysis_plots/2016_zv/cr_vjets_l/fatjet_pt.pdf}
  \includegraphics[width=0.30\textwidth]{analysis_plots/2017_zv/cr_vjets_l/fatjet_pt.pdf}
  \includegraphics[width=0.30\textwidth]{analysis_plots/2018_zv/cr_vjets_l/fatjet_pt.pdf} \\
  \includegraphics[width=0.30\textwidth]{analysis_plots/2016_zv/cr_vjets_l/fatjet_eta.pdf}
  \includegraphics[width=0.30\textwidth]{analysis_plots/2017_zv/cr_vjets_l/fatjet_eta.pdf}
  \includegraphics[width=0.30\textwidth]{analysis_plots/2018_zv/cr_vjets_l/fatjet_eta.pdf} \\
  \includegraphics[width=0.30\textwidth]{analysis_plots/2016_zv/cr_vjets_l/fatjet_m.pdf}
  \includegraphics[width=0.30\textwidth]{analysis_plots/2017_zv/cr_vjets_l/fatjet_m.pdf}
  \includegraphics[width=0.30\textwidth]{analysis_plots/2018_zv/cr_vjets_l/fatjet_m.pdf} \\
  \caption[DY+Jets Control Region: Hadronic boson kinematics in Boosted ZV Channel]%
  {DY+Jets Control Region: Hadronic boson kinematics in Boosted ZV Channel. From Left to Right: 2016,
    2017, and 2018. From Top to Bottom: \( p_T \), \( \eta \), mass \( m \).}%
  \label{fig:zv-cr-vjets-l-fatjet-pt-eta-m}
\end{figure}

\begin{figure}[!ht]
  \centering
  \includegraphics[width=0.30\textwidth]{analysis_plots/2016_zv/cr_vjets_l/vbf_j1_pt.pdf}
  \includegraphics[width=0.30\textwidth]{analysis_plots/2017_zv/cr_vjets_l/vbf_j1_pt.pdf}
  \includegraphics[width=0.30\textwidth]{analysis_plots/2018_zv/cr_vjets_l/vbf_j1_pt.pdf} \\
  \includegraphics[width=0.30\textwidth]{analysis_plots/2016_zv/cr_vjets_l/vbf_j1_eta.pdf}
  \includegraphics[width=0.30\textwidth]{analysis_plots/2017_zv/cr_vjets_l/vbf_j1_eta.pdf}
  \includegraphics[width=0.30\textwidth]{analysis_plots/2018_zv/cr_vjets_l/vbf_j1_eta.pdf} \\
  \includegraphics[width=0.30\textwidth]{analysis_plots/2016_zv/cr_vjets_l/vbf_j1_phi.pdf}
  \includegraphics[width=0.30\textwidth]{analysis_plots/2017_zv/cr_vjets_l/vbf_j1_phi.pdf}
  \includegraphics[width=0.30\textwidth]{analysis_plots/2018_zv/cr_vjets_l/vbf_j1_phi.pdf} \\
  \caption[DY+Jets Control Region: Leading VBS tagged jet kinematics in Boosted ZV Channel]%
  {DY+Jets Control Region: Leading VBS tagged jet kinematics in Boosted ZV Channel. From Left to Right: 2016,
    2017, and 2018. From Top to Bottom: \( p_T \), \( \eta \), \( \phi \).}%
  \label{fig:zv-cr-vjets-l-vbs1-pt-eta-m}
\end{figure}

\begin{figure}[!ht]
  \centering
  \includegraphics[width=0.30\textwidth]{analysis_plots/2016_zv/cr_vjets_l/vbf_j2_pt.pdf}
  \includegraphics[width=0.30\textwidth]{analysis_plots/2017_zv/cr_vjets_l/vbf_j2_pt.pdf}
  \includegraphics[width=0.30\textwidth]{analysis_plots/2018_zv/cr_vjets_l/vbf_j2_pt.pdf} \\
  \includegraphics[width=0.30\textwidth]{analysis_plots/2016_zv/cr_vjets_l/vbf_j2_eta.pdf}
  \includegraphics[width=0.30\textwidth]{analysis_plots/2017_zv/cr_vjets_l/vbf_j2_eta.pdf}
  \includegraphics[width=0.30\textwidth]{analysis_plots/2018_zv/cr_vjets_l/vbf_j2_eta.pdf} \\
  \includegraphics[width=0.30\textwidth]{analysis_plots/2016_zv/cr_vjets_l/vbf_j2_phi.pdf}
  \includegraphics[width=0.30\textwidth]{analysis_plots/2017_zv/cr_vjets_l/vbf_j2_phi.pdf}
  \includegraphics[width=0.30\textwidth]{analysis_plots/2018_zv/cr_vjets_l/vbf_j2_phi.pdf} \\
  \caption[DY+Jets Control Region: Trailing VBS tagged jet kinematics in Boosted ZV Channel]%
  {DY+Jets Control Region: Trailing VBS tagged jet kinematics in Boosted ZV Channel. From Left to Right: 2016,
    2017, and 2018. From Top to Bottom: \( p_T \), \( \eta \), \( \phi \).}%
  \label{fig:zv-cr-vjets-l-vbs2-pt-eta-m}
\end{figure}

\clearpage
\subsection{
  Resolved ZV DY+Jets Control Region
}\label{ch_vbs:resolved-plots}

\begin{figure}[!ht]
  \centering
  \includegraphics[width=0.30\textwidth]{analysis_plots/2016_zjj/cr_vjets_e/lep1_pt.pdf}
  \includegraphics[width=0.30\textwidth]{analysis_plots/2017_zjj/cr_vjets_e/lep1_pt.pdf}
  \includegraphics[width=0.30\textwidth]{analysis_plots/2018_zjj/cr_vjets_e/lep1_pt.pdf} \\
  \includegraphics[width=0.30\textwidth]{analysis_plots/2016_zjj/cr_vjets_e/lep1_eta.pdf}
  \includegraphics[width=0.30\textwidth]{analysis_plots/2017_zjj/cr_vjets_e/lep1_eta.pdf}
  \includegraphics[width=0.30\textwidth]{analysis_plots/2018_zjj/cr_vjets_e/lep1_eta.pdf} \\
  \includegraphics[width=0.30\textwidth]{analysis_plots/2016_zjj/cr_vjets_e/lep1_phi.pdf}
  \includegraphics[width=0.30\textwidth]{analysis_plots/2017_zjj/cr_vjets_e/lep1_phi.pdf}
  \includegraphics[width=0.30\textwidth]{analysis_plots/2018_zjj/cr_vjets_e/lep1_phi.pdf} \\
  \caption[DY+Jets Control Region: Leading electron kinematics in Resolved ZV Channel]%
  {DY+Jets Control Region: Leading electron kinematics in Resolved ZV Channel. From Left to Right: 2016,
    2017, and 2018. From Top to Bottom: \( p_T \), \( \eta \), \( \phi \).}%
  \label{fig:zjj-cr-vjets-e-lep1-pt-eta-phi}
\end{figure}

\begin{figure}[!ht]
  \centering
  \includegraphics[width=0.30\textwidth]{analysis_plots/2016_zjj/cr_vjets_e/lep2_pt.pdf}
  \includegraphics[width=0.30\textwidth]{analysis_plots/2017_zjj/cr_vjets_e/lep2_pt.pdf}
  \includegraphics[width=0.30\textwidth]{analysis_plots/2018_zjj/cr_vjets_e/lep2_pt.pdf} \\
  \includegraphics[width=0.30\textwidth]{analysis_plots/2016_zjj/cr_vjets_e/lep2_eta.pdf}
  \includegraphics[width=0.30\textwidth]{analysis_plots/2017_zjj/cr_vjets_e/lep2_eta.pdf}
  \includegraphics[width=0.30\textwidth]{analysis_plots/2018_zjj/cr_vjets_e/lep2_eta.pdf} \\
  \includegraphics[width=0.30\textwidth]{analysis_plots/2016_zjj/cr_vjets_e/lep2_phi.pdf}
  \includegraphics[width=0.30\textwidth]{analysis_plots/2017_zjj/cr_vjets_e/lep2_phi.pdf}
  \includegraphics[width=0.30\textwidth]{analysis_plots/2018_zjj/cr_vjets_e/lep2_phi.pdf} \\
  \caption[DY+Jets Control Region: Trailing electron kinematics in Resolved ZV Channel]%
  {DY+Jets Control Region: Trailing electron kinematics in Resolved ZV Channel. From Left to Right: 2016,
    2017, and 2018. From Top to Bottom: \( p_T \), \( \eta \), \( \phi \).}%
  \label{fig:zjj-cr-vjets-e-lep2-pt-eta-phi}
\end{figure}

\begin{figure}[!ht]
  \centering
  \includegraphics[width=0.30\textwidth]{analysis_plots/2016_zjj/cr_vjets_m/lep1_pt.pdf}
  \includegraphics[width=0.30\textwidth]{analysis_plots/2017_zjj/cr_vjets_m/lep1_pt.pdf}
  \includegraphics[width=0.30\textwidth]{analysis_plots/2018_zjj/cr_vjets_m/lep1_pt.pdf} \\
  \includegraphics[width=0.30\textwidth]{analysis_plots/2016_zjj/cr_vjets_m/lep1_eta.pdf}
  \includegraphics[width=0.30\textwidth]{analysis_plots/2017_zjj/cr_vjets_m/lep1_eta.pdf}
  \includegraphics[width=0.30\textwidth]{analysis_plots/2018_zjj/cr_vjets_m/lep1_eta.pdf} \\
  \includegraphics[width=0.30\textwidth]{analysis_plots/2016_zjj/cr_vjets_m/lep1_phi.pdf}
  \includegraphics[width=0.30\textwidth]{analysis_plots/2017_zjj/cr_vjets_m/lep1_phi.pdf}
  \includegraphics[width=0.30\textwidth]{analysis_plots/2018_zjj/cr_vjets_m/lep1_phi.pdf} \\
  \caption[DY+Jets Control Region: Leading muon kinematics in Resolved ZV Channel]%
  {DY+Jets Control Region: Leading muon kinematics in Resolved ZV Channel. From Left to Right: 2016,
    2017, and 2018. From Top to Bottom: \( p_T \), \( \eta \), \( \phi \).}%
  \label{fig:zjj-cr-vjets-m-lep1-pt-eta-phi}
\end{figure}

\begin{figure}[!ht]
  \centering
  \includegraphics[width=0.30\textwidth]{analysis_plots/2016_zjj/cr_vjets_m/lep2_pt.pdf}
  \includegraphics[width=0.30\textwidth]{analysis_plots/2017_zjj/cr_vjets_m/lep2_pt.pdf}
  \includegraphics[width=0.30\textwidth]{analysis_plots/2018_zjj/cr_vjets_m/lep2_pt.pdf} \\
  \includegraphics[width=0.30\textwidth]{analysis_plots/2016_zjj/cr_vjets_m/lep2_eta.pdf}
  \includegraphics[width=0.30\textwidth]{analysis_plots/2017_zjj/cr_vjets_m/lep2_eta.pdf}
  \includegraphics[width=0.30\textwidth]{analysis_plots/2018_zjj/cr_vjets_m/lep2_eta.pdf} \\
  \includegraphics[width=0.30\textwidth]{analysis_plots/2016_zjj/cr_vjets_m/lep2_phi.pdf}
  \includegraphics[width=0.30\textwidth]{analysis_plots/2017_zjj/cr_vjets_m/lep2_phi.pdf}
  \includegraphics[width=0.30\textwidth]{analysis_plots/2018_zjj/cr_vjets_m/lep2_phi.pdf} \\
  \caption[DY+Jets Control Region: Trailing muon kinematics in Resolved ZV Channel]%
  {DY+Jets Control Region: Trailing muon kinematics in Resolved ZV Channel. From Left to Right: 2016,
    2017, and 2018. From Top to Bottom: \( p_T \), \( \eta \), \( \phi \).}%
  \label{fig:zjj-cr-vjets-m-lep2-pt-eta-phi}
\end{figure}

\begin{figure}[!ht]
  \centering
  \includegraphics[width=0.30\textwidth]{analysis_plots/2016_zjj/cr_vjets_l/dijet_j1_pt.pdf}
  \includegraphics[width=0.30\textwidth]{analysis_plots/2017_zjj/cr_vjets_l/dijet_j1_pt.pdf}
  \includegraphics[width=0.30\textwidth]{analysis_plots/2018_zjj/cr_vjets_l/dijet_j1_pt.pdf} \\
  \includegraphics[width=0.30\textwidth]{analysis_plots/2016_zjj/cr_vjets_l/dijet_j1_eta.pdf}
  \includegraphics[width=0.30\textwidth]{analysis_plots/2017_zjj/cr_vjets_l/dijet_j1_eta.pdf}
  \includegraphics[width=0.30\textwidth]{analysis_plots/2018_zjj/cr_vjets_l/dijet_j1_eta.pdf} \\
  \includegraphics[width=0.30\textwidth]{analysis_plots/2016_zjj/cr_vjets_l/dijet_m.pdf}
  \includegraphics[width=0.30\textwidth]{analysis_plots/2017_zjj/cr_vjets_l/dijet_m.pdf}
  \includegraphics[width=0.30\textwidth]{analysis_plots/2018_zjj/cr_vjets_l/dijet_m.pdf} \\
  \caption[DY+Jets Control Region: Hadronic boson leading jet kinematics in Resolved ZV Channel]%
  {DY+Jets Control Region: Hadronic boson leading jet kinematics in Resolved ZV Channel. From Left to Right: 2016,
    2017, and 2018. From Top to Bottom: \( p_T \), \( \eta \), invariant mass \( m_{jj} \).}%
  \label{fig:zjj-cr-vjets-l-dijet1-pt-eta-m}
\end{figure}

\begin{figure}[!ht]
  \centering
  \includegraphics[width=0.30\textwidth]{analysis_plots/2016_zjj/cr_vjets_l/dijet_j2_pt.pdf}
  \includegraphics[width=0.30\textwidth]{analysis_plots/2017_zjj/cr_vjets_l/dijet_j2_pt.pdf}
  \includegraphics[width=0.30\textwidth]{analysis_plots/2018_zjj/cr_vjets_l/dijet_j2_pt.pdf} \\
  \includegraphics[width=0.30\textwidth]{analysis_plots/2016_zjj/cr_vjets_l/dijet_j2_eta.pdf}
  \includegraphics[width=0.30\textwidth]{analysis_plots/2017_zjj/cr_vjets_l/dijet_j2_eta.pdf}
  \includegraphics[width=0.30\textwidth]{analysis_plots/2018_zjj/cr_vjets_l/dijet_j2_eta.pdf} \\
  \includegraphics[width=0.30\textwidth]{analysis_plots/2016_zjj/cr_vjets_l/dijet_m.pdf}
  \includegraphics[width=0.30\textwidth]{analysis_plots/2017_zjj/cr_vjets_l/dijet_m.pdf}
  \includegraphics[width=0.30\textwidth]{analysis_plots/2018_zjj/cr_vjets_l/dijet_m.pdf} \\
  \caption[DY+Jets Control Region: Hadronic boson trailing jet kinematics in Resolved ZV Channel]%
  {DY+Jets Control Region: Hadronic boson trailing jet kinematics in Resolved ZV Channel. From Left to Right: 2016,
    2017, and 2018. From Top to Bottom: \( p_T \), \( \eta \), invariant mass \( m_{jj} \).}%
  \label{fig:zjj-cr-vjets-l-dijet2-pt-eta-m}
\end{figure}

\begin{figure}[!ht]
  \centering
  \includegraphics[width=0.30\textwidth]{analysis_plots/2016_zjj/cr_vjets_l/vbf_j1_pt.pdf}
  \includegraphics[width=0.30\textwidth]{analysis_plots/2017_zjj/cr_vjets_l/vbf_j1_pt.pdf}
  \includegraphics[width=0.30\textwidth]{analysis_plots/2018_zjj/cr_vjets_l/vbf_j1_pt.pdf} \\
  \includegraphics[width=0.30\textwidth]{analysis_plots/2016_zjj/cr_vjets_l/vbf_j1_eta.pdf}
  \includegraphics[width=0.30\textwidth]{analysis_plots/2017_zjj/cr_vjets_l/vbf_j1_eta.pdf}
  \includegraphics[width=0.30\textwidth]{analysis_plots/2018_zjj/cr_vjets_l/vbf_j1_eta.pdf} \\
  \includegraphics[width=0.30\textwidth]{analysis_plots/2016_zjj/cr_vjets_l/vbf_j1_phi.pdf}
  \includegraphics[width=0.30\textwidth]{analysis_plots/2017_zjj/cr_vjets_l/vbf_j1_phi.pdf}
  \includegraphics[width=0.30\textwidth]{analysis_plots/2018_zjj/cr_vjets_l/vbf_j1_phi.pdf} \\
  \caption[DY+Jets Control Region: Leading VBS tagged jet kinematics in Resolved ZV Channel]%
  {DY+Jets Control Region: Leading VBS tagged jet kinematics in Resolved ZV Channel. From Left to Right: 2016,
    2017, and 2018. From Top to Bottom: \( p_T \), \( \eta \), \( \phi \).}%
  \label{fig:zjj-cr-vjets-l-vbs1-pt-eta-m}
\end{figure}

\begin{figure}[!ht]
  \centering
  \includegraphics[width=0.30\textwidth]{analysis_plots/2016_zjj/cr_vjets_l/vbf_j2_pt.pdf}
  \includegraphics[width=0.30\textwidth]{analysis_plots/2017_zjj/cr_vjets_l/vbf_j2_pt.pdf}
  \includegraphics[width=0.30\textwidth]{analysis_plots/2018_zjj/cr_vjets_l/vbf_j2_pt.pdf} \\
  \includegraphics[width=0.30\textwidth]{analysis_plots/2016_zjj/cr_vjets_l/vbf_j2_eta.pdf}
  \includegraphics[width=0.30\textwidth]{analysis_plots/2017_zjj/cr_vjets_l/vbf_j2_eta.pdf}
  \includegraphics[width=0.30\textwidth]{analysis_plots/2018_zjj/cr_vjets_l/vbf_j2_eta.pdf} \\
  \includegraphics[width=0.30\textwidth]{analysis_plots/2016_zjj/cr_vjets_l/vbf_j2_phi.pdf}
  \includegraphics[width=0.30\textwidth]{analysis_plots/2017_zjj/cr_vjets_l/vbf_j2_phi.pdf}
  \includegraphics[width=0.30\textwidth]{analysis_plots/2018_zjj/cr_vjets_l/vbf_j2_phi.pdf} \\
  \caption[DY+Jets Control Region: Trailing VBS tagged jet kinematics in Resolved ZV Channel]%
  {DY+Jets Control Region: Trailing VBS tagged jet kinematics in Resolved ZV Channel. From Left to Right: 2016,
    2017, and 2018. From Top to Bottom: \( p_T \), \( \eta \), \( \phi \).}%
  \label{fig:zjj-cr-vjets-l-vbs2-pt-eta-m}
\end{figure}

\clearpage
\section{
  Machine Learning Modeling
 }

Instead of traditional cut-based analysis, we decided to use \gls{MVA} a.k.a. \gls{ML}
technique to build a signal vs background classifier. The main reasoning behind using
a \gls{MVA} technique is so that we can build a model which can learn our
analysis topology from looser selection regions and still let us keep higher statistics
for final measurement.

\subsection{
  Algorithm: Gradient Boosted Decision Tree
}

\gls{BDT} is a \gls{MVA} algorithm with \textit{boosting}
applied sequentially in series on the \gls{MVA} algorithm, and
then taking weighted average to get the improved result,
in this case the boosted\footnote{Not to be confused analysis category
  Boosted ZV, they are unrelated.} \gls{MVA} algorithm is \textit{decision tree}.
A decision tree is a binary tree,
example shown in Figure~\ref{fig:bdt-dt}, at each node
a decision is made using the input variables which
provides the best separation\footnote{defined as,
  \( \frac{1}{2}
  \int \frac{{(\hat{y}_{S}(y) - \hat{y}_{B}(y))}^2}{\hat{y}_{S}(y) + \hat{y}_{B}(y)} dy \),
  where \(\hat{y}_{S}\) and \(\hat{y}_B\) are probability density functions
  of signal and background.}
at that node, this process is repeated
until events at a node reaches below certain threshold, and then depending
on the majority
of the events in the final node, the node is classified
as signal or background.

There are several boosting algorithm, the one discussed here and
used in this analysis is gradient boost.
Gradient boost is an additive model of adding decision trees
in series and with each tree in sequence training on
the adjusted dataset to minimize the \textit{loss-function} with
respect to the previous model. This improves
the model sequentially and it is faster since only a subset of
training dataset is used in subsequent decision trees.
\gls{TMVA} part of \ROOT{} package used in this analysis
for implementing gradient boosted \gls{BDT} has implementation
of binomial log-likelihood loss function for classification~\cite{tmva-manual}.

\begin{figure}
  \centering
  \includegraphics[width=0.5\textwidth]{figures/bdt_pics.pdf}
  \caption[Decision Tree]{Decision Tree.}%
  \label{fig:bdt-dt}
\end{figure}

\subsection{
  Training and Results
}

Two models were trained for boosted and resolved topology and
the training was done using combined MC from all years 2016, 2017, 2018
to benefit from larger statistics see Table~\ref{tab:training-stats},
signal MC ``VBS\_EWK'' was trained against background ``DY + Jets LO'' since
that is dominant background in our analysis.
Each models \gls{BDT} hyper-parameters were tuned to prevent under and over-fitting,
this is done by comparing the signal efficiency
training and testing datasets at three different background rates (1\%, 10\% and 30\%),
if the difference is not more than 5\%, then the hyper-parameters are
acceptable.

Input variables used in each model training were pruned i.e.
input variables which does not impact much on the
performance of the model in terms of \gls{AUC} of the \gls{ROC}
were dropped from the final training models.
Distribution of final input variables
used in training are shown in Figure~\ref{fig:vbs-training-input-zv} and Figure~\ref{fig:vbs-training-input-zjj}.

\begin{table}[!ht]
  \centering
  \caption{Combined dataset Training and Testing Statistics in Signal Region}
  \begin{tabular}{lllll}%
    \toprule
             &            & \multicolumn{3}{c}{Number of Events}                    \\
    \cmidrule(lr){3-5}
    Channel  & Dataset    & Training                             & Testing & Total  \\
    \midrule
    Boosted  & Signal     & 7404                                 & 7405    & 14809  \\
    Boosted  & Background & 46991                                & 46991   & 93982  \\
    \midrule
    Resolved & Signal     & 23425                                & 23425   & 46850  \\
    Resolved & Background & 209368                               & 209368  & 418736 \\
    \bottomrule
  \end{tabular}\label{tab:training-stats}
\end{table}

\begin{figure}[!ht]
  \centering
  \includegraphics[width=0.23\textwidth]{analysis_plots/tmva_plots/zv_BDTG14_dibos_m.pdf}
  \includegraphics[width=0.23\textwidth]{analysis_plots/2016_zv/cr_vjets_l/vv_m.pdf}
  \includegraphics[width=0.23\textwidth]{analysis_plots/2017_zv/cr_vjets_l/vv_m.pdf}
  \includegraphics[width=0.23\textwidth]{analysis_plots/2018_zv/cr_vjets_l/vv_m.pdf}\\
  \includegraphics[width=0.23\textwidth]{analysis_plots/tmva_plots/zv_BDTG14_vbf_m.pdf}
  \includegraphics[width=0.23\textwidth]{analysis_plots/2016_zv/cr_vjets_l/vbf_jj_m.pdf}
  \includegraphics[width=0.23\textwidth]{analysis_plots/2017_zv/cr_vjets_l/vbf_jj_m.pdf}
  \includegraphics[width=0.23\textwidth]{analysis_plots/2018_zv/cr_vjets_l/vbf_jj_m.pdf}\\
  \includegraphics[width=0.23\textwidth]{analysis_plots/tmva_plots/zv_BDTG14_vbf1_AK4_qgid.pdf}
  \includegraphics[width=0.23\textwidth]{analysis_plots/2016_zv/cr_vjets_l/vbf_j1_qgid.pdf}
  \includegraphics[width=0.23\textwidth]{analysis_plots/2017_zv/cr_vjets_l/vbf_j1_qgid.pdf}
  \includegraphics[width=0.23\textwidth]{analysis_plots/2018_zv/cr_vjets_l/vbf_j1_qgid.pdf}\\
  \includegraphics[width=0.23\textwidth]{analysis_plots/tmva_plots/zv_BDTG14_vbf2_AK4_qgid.pdf}
  \includegraphics[width=0.23\textwidth]{analysis_plots/2016_zv/cr_vjets_l/vbf_j2_qgid.pdf}
  \includegraphics[width=0.23\textwidth]{analysis_plots/2017_zv/cr_vjets_l/vbf_j2_qgid.pdf}
  \includegraphics[width=0.23\textwidth]{analysis_plots/2018_zv/cr_vjets_l/vbf_j2_qgid.pdf}\\
  \includegraphics[width=0.23\textwidth]{analysis_plots/tmva_plots/zv_BDTG14_zeppLep_deta.pdf}
  \includegraphics[width=0.23\textwidth]{analysis_plots/2016_zv/cr_vjets_l/zeppenfeld_lep_deta.pdf}
  \includegraphics[width=0.23\textwidth]{analysis_plots/2017_zv/cr_vjets_l/zeppenfeld_lep_deta.pdf}
  \includegraphics[width=0.23\textwidth]{analysis_plots/2018_zv/cr_vjets_l/zeppenfeld_lep_deta.pdf}\\
  \caption[Inputs Variables for training Boosted ZV BDT Classifier]%
  {Inputs Variables for training Boosted ZV BDT Classifier.
    Left to Right: Shape comparison of signal and background (combined dataset), control region
    kinematic distribution in 2016, 2017 and 2018 dataset.
    Top to Bottom: Diboson invariant mass, VBS tagged jets invariant mass, \gls{QGL} of leading
    VBS tagged jet, \gls{QGL} of trailing VBS tagged jet, Zeppenfeld variable of leptonic boson.}%
  \label{fig:vbs-training-input-zv}
\end{figure}


%\begin{figure}[!ht]
%  \centering
%  \includegraphics[width=0.4\textwidth]{analysis_plots/tmva_plots/zv_BDTG14_dibos_m.pdf}
%  \includegraphics[width=0.4\textwidth]{analysis_plots/tmva_plots/zv_BDTG14_vbf_m.pdf} \\
%  \includegraphics[width=0.4\textwidth]{analysis_plots/tmva_plots/zv_BDTG14_vbf1_AK4_qgid.pdf}
%  \includegraphics[width=0.4\textwidth]{analysis_plots/tmva_plots/zv_BDTG14_vbf2_AK4_qgid.pdf} \\
%  \includegraphics[width=0.4\textwidth]{analysis_plots/tmva_plots/zv_BDTG14_zeppLep_deta.pdf}
%  \caption[Inputs Variables for training Boosted ZV BDT Classifier]%
%  {Inputs Variables (combined for Run2 MC) for training Boosted ZV BDT Classifier.
%    From top: Diboson invariant mass, VBS tagged jets invariant mass, \gls{QGL} of leading
%    VBS tagged jet, \gls{QGL} of trailing VBS tagged jet, Zeppenfeld variable of leptonic boson.}%
%  \label{fig:vbs-training-input-zv}
%\end{figure}

\begin{figure}[!ht]
  \centering
  \includegraphics[width=0.24\textwidth]{analysis_plots/tmva_plots/zjj_BDTG14_dibos_m.pdf}
  \includegraphics[width=0.24\textwidth]{analysis_plots/2016_zjj/cr_vjets_l/vv_m.pdf}
  \includegraphics[width=0.24\textwidth]{analysis_plots/2017_zjj/cr_vjets_l/vv_m.pdf}
  \includegraphics[width=0.24\textwidth]{analysis_plots/2018_zjj/cr_vjets_l/vv_m.pdf} \\
  \includegraphics[width=0.24\textwidth]{analysis_plots/tmva_plots/zjj_BDTG14_vbf_m.pdf}
  \includegraphics[width=0.24\textwidth]{analysis_plots/2016_zjj/cr_vjets_l/vbf_jj_m.pdf}
  \includegraphics[width=0.24\textwidth]{analysis_plots/2017_zjj/cr_vjets_l/vbf_jj_m.pdf}
  \includegraphics[width=0.24\textwidth]{analysis_plots/2018_zjj/cr_vjets_l/vbf_jj_m.pdf}\\
  \includegraphics[width=0.24\textwidth]{analysis_plots/tmva_plots/zjj_BDTG14_ht_resolved.pdf}
  \includegraphics[draft,width=0.24\textwidth]{fixme.pdf}
  \includegraphics[draft,width=0.24\textwidth]{fixme.pdf}
  \includegraphics[draft,width=0.24\textwidth]{fixme.pdf}\\
  \includegraphics[width=0.24\textwidth]{analysis_plots/tmva_plots/zjj_BDTG14_lep2_eta.pdf}
  \includegraphics[width=0.24\textwidth]{analysis_plots/2016_zjj/cr_vjets_l/lep2_eta.pdf}
  \includegraphics[width=0.24\textwidth]{analysis_plots/2017_zjj/cr_vjets_l/lep2_eta.pdf}
  \includegraphics[width=0.24\textwidth]{analysis_plots/2018_zjj/cr_vjets_l/lep2_eta.pdf}
  \caption[Inputs Variables for training Resolved ZV BDT Classifier]%
  {Inputs Variables for training Resolved ZV BDT Classifier.
    Left to Right: Shape comparison of signal and background (combined dataset), control region
    kinematic distribution in 2016, 2017 and 2018 dataset.
    Top to Bottom: Diboson invariant mass, VBS tagged jets invariant mass,
    HT\textsuperscript{*} (\( p_{T} \) sum of jets), trailing lepton \( \eta \).}%
  \label{fig:vbs-training-input-zjj}
\end{figure}

\begin{figure}[!ht]
  \centering
  \includegraphics[width=0.24\textwidth]{analysis_plots/tmva_plots/zjj_BDTG14_vbf1_AK4_qgid.pdf}
  \includegraphics[width=0.24\textwidth]{analysis_plots/2016_zjj/cr_vjets_l/vbf_j1_qgid.pdf}
  \includegraphics[width=0.24\textwidth]{analysis_plots/2017_zjj/cr_vjets_l/vbf_j1_qgid.pdf}
  \includegraphics[width=0.24\textwidth]{analysis_plots/2018_zjj/cr_vjets_l/vbf_j1_qgid.pdf} \\
  \includegraphics[width=0.24\textwidth]{analysis_plots/tmva_plots/zjj_BDTG14_vbf2_AK4_qgid.pdf}
  \includegraphics[width=0.24\textwidth]{analysis_plots/2016_zjj/cr_vjets_l/vbf_j2_qgid.pdf}
  \includegraphics[width=0.24\textwidth]{analysis_plots/2017_zjj/cr_vjets_l/vbf_j2_qgid.pdf}
  \includegraphics[width=0.24\textwidth]{analysis_plots/2018_zjj/cr_vjets_l/vbf_j2_qgid.pdf} \\
  \includegraphics[width=0.24\textwidth]{analysis_plots/tmva_plots/zjj_BDTG14_zeppLep_deta.pdf}
  \includegraphics[width=0.24\textwidth]{analysis_plots/2016_zjj/cr_vjets_l/zeppenfeld_lep_deta.pdf}
  \includegraphics[width=0.24\textwidth]{analysis_plots/2017_zjj/cr_vjets_l/zeppenfeld_lep_deta.pdf}
  \includegraphics[width=0.24\textwidth]{analysis_plots/2018_zjj/cr_vjets_l/zeppenfeld_lep_deta.pdf}
  \caption[Inputs Variables for training Resolved ZV BDT Classifier]%
  {Inputs Variables for training Resolved ZV BDT Classifier.
    Left to Right: Shape comparison of signal and background (combined dataset), control region
    kinematic distribution in 2016, 2017 and 2018 dataset.
    Top to Bottom:\gls{QGL} of leading VBS tagged jet,
    \gls{QGL} of trailing VBS tagged jet, Zeppenfeld variable of leptonic boson.}%
  \label{fig:vbs-training-input-zjj-1}
\end{figure}



%\begin{figure}[!ht]
%  \centering
%  \includegraphics[width=0.32\textwidth]{analysis_plots/tmva_plots/zjj_BDTG14_dibos_m.pdf}
%  \includegraphics[width=0.32\textwidth]{analysis_plots/tmva_plots/zjj_BDTG14_vbf_m.pdf} \\
%  \includegraphics[width=0.32\textwidth]{analysis_plots/tmva_plots/zjj_BDTG14_ht_resolved.pdf}
%  \includegraphics[width=0.32\textwidth]{analysis_plots/tmva_plots/zjj_BDTG14_lep2_eta.pdf} \\
%  \includegraphics[width=0.32\textwidth]{analysis_plots/tmva_plots/zjj_BDTG14_vbf1_AK4_qgid.pdf}
%  \includegraphics[width=0.32\textwidth]{analysis_plots/tmva_plots/zjj_BDTG14_vbf2_AK4_qgid.pdf}
%  \includegraphics[width=0.32\textwidth]{analysis_plots/tmva_plots/zjj_BDTG14_zeppLep_deta.pdf}
%  \caption[Inputs Variables for training Resolved ZV BDT Classifier]%
%  {Inputs Variables (combined for Run2 MC) for training Resolved ZV BDT Classifier.
%    From top: Diboson invariant mass, VBS tagged jets invariant mass,
%    HT\textsuperscript{*} (\( p_{T} \) sum of jets), trailing lepton \( \eta \),
%    \gls{QGL} of leading VBS tagged jet,
%    \gls{QGL} of trailing VBS tagged jet, Zeppenfeld variable of leptonic boson.}%
%  \label{fig:vbs-training-input-zjj}
%\end{figure}

After training \gls{TMVA} \gls{BDT} evaluates input variables
and ranks them in terms of importance and separation
they provide in classification is listed in Table~\ref{tab:training-input-rank},
and the correlation matrix of variable is show in Figure~\ref{fig:vbs-training-correlation}.

\begin{table}[!ht]
  \centering
  \caption{Training Input Variable Ranking}
  \begin{tabular}{lllll}%
    \toprule
    Channel & Variable                   & Variable Name        & Importance & Separation \\
    \midrule
    \multirow{5}{*}{Boosted}
            & \( M_{JJ}^{VBS} \)         & \verb|vbf_m|         & 0.2496     & 0.1348     \\
            & Zeppenfeld \( Z_{V}^{*} \) & \verb|zeppLep_deta|  & 0.2396     & 0.1116     \\
            & QGL \( j_{1}^{VBS} \)      & \verb|vbf2_AK4_qgid| & 0.1889     & 0.02413    \\
            & QGL \( j_{2}^{VBS} \)      & \verb|vbf1_AK4_qgid| & 0.1780     & 0.02330    \\
            & \( M_{VV} \)               & \verb|dibos_m|       & 0.1439     & 0.005308   \\
    \midrule
    \multirow{7}{*}{Resolved}
            & Zeppenfeld \( Z_{V}^{*} \) & \verb|zeppLep_deta|  & 0.1955     & 0.1219     \\
            & \( M_{JJ}^{VBS} \)         & \verb|vbf_m|         & 0.1822     & 0.07998    \\
            & HT\textsuperscript{*}      & \verb|ht_resolved|   & 0.1693     & 0.04201    \\
            & QGL \( j_{1}^{VBS} \)      & \verb|vbf2_AK4_qgid| & 0.1403     & 0.02159    \\
            & QGL \( j_{2}^{VBS} \)      & \verb|vbf1_AK4_qgid| & 0.1341     & 0.03235    \\
            & \( M_{VV} \)               & \verb|dibos_m|       & 0.09098    & 0.01112    \\
            & \( \eta_{lep2} \)          & \verb|lep2_eta|      & 0.08760    & 0.01755    \\
    \bottomrule
  \end{tabular}\label{tab:training-input-rank}
\end{table}

\begin{figure}[!ht]
  \centering
  \includegraphics[width=0.4\textwidth]{analysis_plots/tmva_plots/zv_BDTG14_CorrelationMatrixS.pdf}
  \includegraphics[width=0.4\textwidth]{analysis_plots/tmva_plots/zjj_BDTG14_CorrelationMatrixS.pdf} \\
  \includegraphics[width=0.4\textwidth]{analysis_plots/tmva_plots/zv_BDTG14_CorrelationMatrixB.pdf}
  \includegraphics[width=0.4\textwidth]{analysis_plots/tmva_plots/zjj_BDTG14_CorrelationMatrixB.pdf} \\
  \caption[Correlation Matrix for Signal and Background]%
  {Correlation Matrix for Signal and Background. From Left to Right: Boosted, Resolved.
    From Top to Bottom: Signal, Background}%
  \label{fig:vbs-training-correlation}
\end{figure}

The under and over-fitting of trained model is checked by \gls{K-S} test
and \gls{ROC} curves comparison between training and testing datasets.
If the tests are not acceptable, then the training is redone with adjusted parameters.
The Table~\ref{tab:training-tmva-config} lists the final
training configuration in the \gls{TMVA} framework for
for boosted and resolved models.
The Figure~\ref{fig:vbs-training-score} show \gls{MVA} score and \gls{ROC} curves
of the BDT models. Boosted trained model \gls{MVA} has 78 \% and resolved
has 79 \% of \gls{AUC}.

\begin{table}[!ht]
  \centering
  \caption{TMVA configuration of trained models}
  \begin{tabular}{lc}%
    \toprule
                            & Boosted and Resolved ZV \\
    \midrule\relax
    \verb|BoostType|        & \verb|Grad|             \\
    \verb|NTrees|           & 800                     \\
    \verb|MaxDepth|         & 3                       \\
    \verb|MinNodeSize|      & 3\%                     \\
    \verb|nCuts|            & 100                     \\
    \verb|Shrinkage|        & 0.010                   \\
    \verb|UseBaggedBoost|   & \verb|True|             \\
    \verb|BaggedSampleFrac| & 0.6                     \\
    \bottomrule
  \end{tabular}\label{tab:training-tmva-config}
\end{table}

\begin{figure}[!ht]
  \centering
  \includegraphics[width=0.4\textwidth]{analysis_plots/tmva_plots/zv_BDTG14.pdf}
  \includegraphics[width=0.4\textwidth]{analysis_plots/tmva_plots/zjj_BDTG14.pdf} \\
  \includegraphics[width=0.4\textwidth]{analysis_plots/tmva_plots/zv_BDTG14_roc.pdf}
  \includegraphics[width=0.4\textwidth]{analysis_plots/tmva_plots/zjj_BDTG14_roc.pdf}
  \caption[MVA Score ROC Curve]%
  {From Left to Right: Boosted, Resolved. Top to Bottom: MVA Score of BDT models,
    ROC Curves.}%
  \label{fig:vbs-training-score}
\end{figure}

\clearpage
\subsection{
  MVA Score Inference
}

Using the training models for boosted and resolved ZV, the \gls{MVA} score
is inferred for the complete dataset of data and \gls{MC}.
Then the binning of \gls{MVA} score is derived such that
the signal yield is almost equal in all the bins in signal region.
\gls{MVA} score distribution for control region with
data is shown in Figure~\ref{fig:zv-cr-l-mva-score} and~\ref{fig:zjj-cr-l-mva-score},
and blinded \gls{MVA} score for signal region
is shown in Figure~\ref{fig:zv-sr-l-mva-score}
and Figure~\ref{fig:zjj-cr-l-mva-score}.

\begin{figure}[!ht]
  \centering
  \includegraphics[width=0.32\textwidth]{analysis_plots/2016_zv/cr_vjets_l/mva_score_zv_var2_log.pdf}
  \includegraphics[width=0.32\textwidth]{analysis_plots/2017_zv/cr_vjets_l/mva_score_zv_var2_log.pdf}
  \includegraphics[width=0.32\textwidth]{analysis_plots/2018_zv/cr_vjets_l/mva_score_zv_var2_log.pdf} \\
  \caption[MVA Score in Control Region for Boosted ZV Channel]%
  {MVA Score in Control Region for Boosted ZV Channel.}%
  \label{fig:zv-cr-l-mva-score}
\end{figure}

\begin{figure}[!ht]
  \centering
  \includegraphics[width=0.32\textwidth]{analysis_plots/2016_zjj/cr_vjets_l/mva_score_zjj_var2_log.pdf}
  \includegraphics[width=0.32\textwidth]{analysis_plots/2017_zjj/cr_vjets_l/mva_score_zjj_var2_log.pdf}
  \includegraphics[width=0.32\textwidth]{analysis_plots/2018_zjj/cr_vjets_l/mva_score_zjj_var2_log.pdf} \\
  \caption[MVA Score in Control Region for Resolved ZV Channel]%
  {MVA Score in Control Region for Resolved ZV Channel.}%
  \label{fig:zjj-cr-l-mva-score}
\end{figure}

\begin{figure}[!ht]
  \centering
  \includegraphics[width=0.32\textwidth]{analysis_plots/2016_zv/sr_l/mva_score_zv_var2_log.pdf}
  \includegraphics[width=0.32\textwidth]{analysis_plots/2017_zv/sr_l/mva_score_zv_var2_log.pdf}
  \includegraphics[width=0.32\textwidth]{analysis_plots/2018_zv/sr_l/mva_score_zv_var2_log.pdf} \\
  \caption[MVA Score in Signal Region for Boosted ZV Channel]%
  {MVA Score in Signal Region for Boosted ZV Channel.}%
  \label{fig:zv-sr-l-mva-score}
\end{figure}

\begin{figure}[!ht]
  \centering
  \includegraphics[width=0.32\textwidth]{analysis_plots/2016_zjj/sr_l/mva_score_zjj_var2_log.pdf}
  \includegraphics[width=0.32\textwidth]{analysis_plots/2017_zjj/sr_l/mva_score_zjj_var2_log.pdf}
  \includegraphics[width=0.32\textwidth]{analysis_plots/2018_zjj/sr_l/mva_score_zjj_var2_log.pdf} \\
  \caption[MVA Score in Signal Region for Resolved ZV Channel]%
  {MVA Score in Signal Region for Resolved ZV Channel.}%
  \label{fig:zjj-sr-l-mva-score}
\end{figure}


\section{
  Measurement
 }

To measure our signal,
the \gls{MVA} score in signal region is used as the observable to
calculate expected significance and study impact of systematics
using impact plots.

\verb|CombineLimit| is command line tool which takes input
configuration as plain text file and can perform
standard high energy physics statistical analysis. \verb|CombineLimit|
is built on \verb|RooFit| and \verb|RooStats| packages which
are also part of ROOT package~\cite{Verkerke2003,Moneta2010}.

\subsection{
  Statistical and Systematic Uncertainties
}

\gls{MC} statistical uncertainties for the bins with more than 10 events
are taken into account using ``Barlow–-Beeston lite'' method,
and using Poisson \gls{p.d.f.} otherwise~\cite{Barlow1993}.

Systematic uncertainties
for correction of multiplicative type such as integrated luminosity, efficiency, etc.,
are treated with most common log-normal distribution model,
i.e.~multiplying with factor \( \kappa \) is equivalent to
\( +1 \sigma \) variation, and
multiplying with \( 1/\kappa \) to get \( -1\sigma \) variation. For relative
uncertainty \(\Delta x / x \), the \( \kappa \) can be set as \( 1 + \Delta x / x \).
For example, 2.5\% uncertainty with log-normal model on integrated luminosity,
\( \kappa \) will be 1.025. Uncertainties set this way
are referred to as ``flat'' systematic uncertainties.

For systematic uncertainties such as \gls{JES}, observable
used in the measurement is also calculated with \gls{JES} \( +1\sigma \)
and \( -1\sigma \) scale factor separately, and the variation up and down
value are directly taken from the histogram bin, these type of uncertainties
are referred to as ``shape'' systematic uncertainties.

Following are the systematic uncertainties used in this analysis,

\begin{itemize}
  \item \textbf{Integrated Luminosity}: 2.5\%
        uncertainty on all 2016, 2017, and 2018
        year datasets.
  \item \textbf{L1 Prefire Correction}: Shape systematics for 2016 and 2017
        year.
  \item \textbf{Lepton Reconstruction and Identification Efficiency}: 1\%
        uncertainty for electron and muons.
  \item \textbf{Lepton momentum scale}: Shape systematics for electron
        and muon momentum scale, shapes are obtained by varying \(p_T\) 3\% up and
        down in selection.
  \item \textbf{\gls{JES} }: Shape systematics
        obtained by varying jet \( p_T \) with \gls{JES} up and down,
        for all the years dataset. The uncertainties
        are treated separately depending on the source~\cite{CMS-DP-2020-019}.
  \item \textbf{PU Reweighting}: Shape systematics
        obtained by varying normalization up and down for all the years dataset.
  \item \textbf{Renormalization and factorization scale}:
        Renormalization and factorization scale uncertainties
        are calculated for VBS\_EWK and VBS\_QCD \gls{MC} datasets,
        as prescribed by \gls{CMS} the renormalization and factorization
        scale factor are varied up and down by the factor of 2,
        and the envelope of the all the variation is taken as \( 1\sigma \)
        shape systematics.
  \item \textbf{PDF systematics}: \gls{PDF} uncertainties
        are obtained as uniform shape systematic by reweighting \gls{MC} events cross-section
        with \gls{PDF} set.
\end{itemize}

\subsection{
  Significance
}

Signal significance (\( Z \)) is
the quantity used to report if there is excess of signal events (\( s \)) in
presence of background events (\( b \)) by testing alternate (\( n = \mu s + b \)) hypothesis
against null (\( n = b \)) hypothesis, where \( \mu \) is the signal strength.

Signal significance (\( Z \)) in the asymptotic limit
is calculated using profile likelihood ratio as the test statistics~\cite{Cowan2010},

\begin{equation}
  Z = \sqrt{q_0}
\end{equation}

\begin{equation}
  q_{\mu} = - 2 \log \frac{L(data | \mu,\hat{\theta}_{\mu})}{L(data | \hat{\mu},\hat{\theta})},\quad 0 \le \mu \le \hat{\mu}
\end{equation}

where \( \hat{\mu} \) is the maximum likelihood estimate of \( \mu \),
and \( \hat{\theta}_{\mu} \) are the nuisance parameters.

As per \gls{LHC} statistical recommendation~\cite{CMS-NOTE-2011-005},
likelihood \( L(data | \mu,\hat{\theta}_{\mu}) \)
is constructed as,

\begin{equation}
  L(data | \mu,\hat{\theta}_{\mu}) = \text{Poisson}(data|\mu s(\hat{\theta}) + b(\hat{\theta})) \cdot \pi(\hat{\theta})
\end{equation}

where \( \pi(\hat{\theta}) \) is the \gls{p.d.f.} for nuisance parameter.

For calculating expected significance for \( \mu = 1 \),
the observed data is not used,
instead \( n \) is set equal to total \gls{MC} events.
The expected significance calculated for
each category per year dataset, for combined dataset per
category and for combined dataset and category is listed
in Table~\ref{tab:vbs-significance}.

\begin{table}[!ht]
  \centering
  \caption{Expected Signal Significance}
  \begin{tabular}{lllll}%
    \toprule
    Channel  & 2016 & 2017 & 2018 & Combined \\
    \midrule
    Boosted  & 0.64 & 0.66 & 0.95 & 1.33     \\
    Resolved & 0.40 & 0.37 & 0.56 & 0.78     \\
    \midrule
    Combined &      &      &      & 1.52     \\
    \bottomrule
  \end{tabular}\label{tab:vbs-significance}
\end{table}

\clearpage
\subsection{
  Impact Plots
}

Impact plots are way to visualize
which nuisance parameter \( \theta \) have
largest effect on parameter of interest \( \mu \) and
measure correlation from the direction of \( +1 \sigma \) (correlated)
and \( -1 \sigma \) (anti-correlated). Figure~\ref{fig:vbs-impact-plots-page1},
~\ref{fig:vbs-impact-plots-page2} shows impacts
for top 60 nuisance parameter, and
Figure~\ref{fig:vbs-impact-plots-page3},
~\ref{fig:vbs-impact-plots-page4} in the Appendix~\ref{app2:impact-plots}
for the rest of nuisance parameter.

\begin{figure}[!ht]
  \centering
  \includegraphics[width=\textwidth,page=1]{analysis_plots/impact_plots/impacts_datacard_run2_z.pdf}
  \caption[Impact Plots]%
  {Impact Plots}%
  \label{fig:vbs-impact-plots-page1}
\end{figure}

\begin{figure}[!ht]
  \centering
  \includegraphics[width=\textwidth,page=2]{analysis_plots/impact_plots/impacts_datacard_run2_z.pdf}
  \caption[Impact Plots]%
  {Impact Plots}%
  \label{fig:vbs-impact-plots-page2}
\end{figure}


\clearpage
\subsection{
  Postfit Plots
}

Postfit distribution in signal region using Asimov dataset
for Boosted ZV are shown in Figure~\ref{fig:zv-cr-l-mva-score}
and for Resolved ZV are shown in Figure~\ref{fig:zjj-sr-l-mva-score-postfit}.
\gls{MC} processes overall prefit and postfit (signal + background and background-only)
values in control and signal region are listed in Table~\ref{tab:fit-values-zv}
for Boosted ZV and in Table~\ref{tab:fit-values-zjj} for Resolved ZV category.

\begin{figure}[!ht]
  \centering
  \includegraphics[width=0.32\textwidth]{analysis_plots/2016_zv.sr_l_postfit/sr_l_postfit/mva_score_zv_var2_log.pdf}
  \includegraphics[width=0.32\textwidth]{analysis_plots/2017_zv.sr_l_postfit/sr_l_postfit/mva_score_zv_var2_log.pdf}
  \includegraphics[width=0.32\textwidth]{analysis_plots/2018_zv.sr_l_postfit/sr_l_postfit/mva_score_zv_var2_log.pdf} \\
  \caption[MVA Score postfit in Signal Region for Boosted ZV Channel]%
  {(Asimov Data) MVA Score postfit in Signal Region for Boosted ZV Channel.}%
  \label{fig:zv-sr-l-mva-score-postfit}
\end{figure}

\begin{figure}[!ht]
  \centering
  \includegraphics[width=0.32\textwidth]{analysis_plots/2016_zjj.sr_l_postfit/sr_l_postfit/mva_score_zjj_var2_log.pdf}
  \includegraphics[width=0.32\textwidth]{analysis_plots/2017_zjj.sr_l_postfit/sr_l_postfit/mva_score_zjj_var2_log.pdf}
  \includegraphics[width=0.32\textwidth]{analysis_plots/2018_zjj.sr_l_postfit/sr_l_postfit/mva_score_zjj_var2_log.pdf} \\
  \caption[MVA Score postfit in Signal Region for Resolved ZV Channel]%
  {(Asimov Data) MVA Score postfit in Signal Region for Resolved ZV Channel. (Asimov Data)}%
  \label{fig:zjj-sr-l-mva-score-postfit}
\end{figure}

\begin{table}
  \centering
  \caption{Prefit and postfit yields of MC processes}
  \begin{tabular}{lllll}
    \toprule
    Region & Process  & Pre-fit                  & S+B Fit                  & B-Only Fit               \\
    \midrule
    \multicolumn{5}{c}{Boosted ZV 2016}                                                                \\
    \midrule
    CR     & DY+Jets  & 379.258 \( \pm \) 21.758 & 365.946 \( \pm \) 21.633 & 371.954 \( \pm \) 19.224 \\
    CR     & Top      & 32.874 \( \pm \) 2.048   & 33.083 \( \pm \) 1.893   & 33.133 \( \pm \) 2.033   \\
    CR     & VBS\_EWK & 5.230 \( \pm \) 0.295    & 5.263 \( \pm \) 7.937    & 0.000 \( \pm \) 0.000    \\
    CR     & VBS\_QCD & 38.659 \( \pm \) 2.269   & 38.903 \( \pm \) 2.098   & 38.964 \( \pm \) 2.229   \\
    SR     & DY+Jets  & 414.424 \( \pm \) 47.960 & 401.166 \( \pm \) 33.571 & 409.919 \( \pm \) 35.398 \\
    SR     & Top      & 37.557 \( \pm \) 8.904   & 38.063 \( \pm \) 7.913   & 38.377 \( \pm \) 8.120   \\
    SR     & VBS\_EWK & 11.280 \( \pm \) 1.263   & 11.547 \( \pm \) 17.344  & 0.000 \( \pm \) 0.000    \\
    SR     & VBS\_QCD & 73.244 \( \pm \) 6.401   & 74.917 \( \pm \) 6.040   & 75.819 \( \pm \) 5.618   \\
    \midrule
    \multicolumn{5}{c}{Boosted ZV 2017}                                                                \\
    \midrule
    CR     & DY+Jets  & 482.465 \( \pm \) 27.558 & 477.203 \( \pm \) 23.379 & 484.032 \( \pm \) 20.462 \\
    CR     & Top      & 25.458 \( \pm \) 2.499   & 25.382 \( \pm \) 2.326   & 25.393 \( \pm \) 2.440   \\
    CR     & VBS\_EWK & 5.584 \( \pm \) 0.315    & 5.598 \( \pm \) 8.314    & 0.000 \( \pm \) 0.000    \\
    CR     & VBS\_QCD & 42.475 \( \pm \) 2.428   & 42.564 \( \pm \) 2.257   & 42.617 \( \pm \) 2.366   \\
    SR     & DY+Jets  & 517.728 \( \pm \) 35.203 & 526.548 \( \pm \) 29.418 & 537.276 \( \pm \) 22.960 \\
    SR     & Top      & 42.445 \( \pm \) 4.879   & 43.314 \( \pm \) 5.419   & 43.618 \( \pm \) 5.213   \\
    SR     & VBS\_EWK & 13.635 \( \pm \) 1.288   & 13.958 \( \pm \) 20.084  & 0.000 \( \pm \) 0.000    \\
    SR     & VBS\_QCD & 83.087 \( \pm \) 6.196   & 85.369 \( \pm \) 6.421   & 86.344 \( \pm \) 6.211   \\
    \midrule
    \multicolumn{5}{c}{Boosted ZV 2018}                                                                \\
    \midrule
    CR     & DY+Jets  & 704.184 \( \pm \) 37.672 & 657.159 \( \pm \) 29.757 & 668.863 \( \pm \) 27.491 \\
    CR     & Top      & 47.431 \( \pm \) 2.598   & 48.020 \( \pm \) 2.506   & 48.115 \( \pm \) 2.524   \\
    CR     & VBS\_EWK & 10.045 \( \pm \) 0.535   & 10.167 \( \pm \) 11.167  & 0.000 \( \pm \) 0.000    \\
    CR     & VBS\_QCD & 68.708 \( \pm \) 3.720   & 69.558 \( \pm \) 3.569   & 69.698 \( \pm \) 3.631   \\
    SR     & DY+Jets  & 746.410 \( \pm \) 54.277 & 691.760 \( \pm \) 33.952 & 710.128 \( \pm \) 25.912 \\
    SR     & Top      & 64.774 \( \pm \) 5.066   & 65.157 \( \pm \) 4.307   & 65.832 \( \pm \) 4.241   \\
    SR     & VBS\_EWK & 23.232 \( \pm \) 1.676   & 23.579 \( \pm \) 25.419  & 0.000 \( \pm \) 0.000    \\
    SR     & VBS\_QCD & 131.658 \( \pm \) 9.406  & 133.540 \( \pm \) 9.249  & 135.256 \( \pm \) 8.750  \\
    \bottomrule
  \end{tabular}\label{tab:fit-values-zv}
\end{table}


\begin{table}
  \centering
  \caption{Prefit and postfit yields of MC processes}
  \resizebox{\textwidth}{!}{%
    \begin{tabular}{lllll}
      \toprule
      Region & Process  & Pre-fit                      & S+B Fit                     & B-Only Fit                  \\
      \midrule
      \multicolumn{5}{c}{Resolved ZV 2016}                                                                         \\
      \midrule
      CR     & DY+Jets  & 379.258 \( \pm \) 21.758     & 365.242 \( \pm \) 22.024    & 371.274 \( \pm \) 19.206    \\
      CR     & Top      & 32.874 \( \pm \) 2.048       & 33.210 \( \pm \) 1.899      & 33.266 \( \pm \) 2.042      \\
      CR     & VBS\_EWK & 5.230 \( \pm \) 0.295        & 5.282 \( \pm \) 8.611       & 0.000 \( \pm \) 0.000       \\
      CR     & VBS\_QCD & 38.659 \( \pm \) 2.269       & 39.050 \( \pm \) 2.104      & 39.119 \( \pm \) 2.238      \\
      SR     & DY+Jets  & 414.424 \( \pm \) 47.924     & 399.587 \( \pm \) 34.001    & 408.251 \( \pm \) 36.425    \\
      SR     & Top      & 37.557 \( \pm \) 8.913       & 38.257 \( \pm \) 7.930      & 38.557 \( \pm \) 8.536      \\
      SR     & VBS\_EWK & 11.280 \( \pm \) 1.314       & 11.617 \( \pm \) 18.849     & 0.000 \( \pm \) 0.000       \\
      SR     & VBS\_QCD & 73.244 \( \pm \) 6.371       & 75.326 \( \pm \) 6.052      & 76.303 \( \pm \) 5.618      \\
      \midrule
      \multicolumn{5}{c}{Resolved ZV 2017}                                                                         \\
      \midrule
      CR     & DY+Jets  & 9269.820 \( \pm \) 528.262   & 8808.212 \( \pm \) 127.281  & 8822.983 \( \pm \) 115.454  \\
      CR     & Top      & 732.264 \( \pm \) 41.867     & 758.852 \( \pm \) 42.171    & 760.721 \( \pm \) 42.469    \\
      CR     & VBS\_EWK & 16.719 \( \pm \) 0.942       & 17.307 \( \pm \) 36.196     & 0.000 \( \pm \) 0.000       \\
      CR     & VBS\_QCD & 212.000 \( \pm \) 11.956     & 219.481 \( \pm \) 12.029    & 220.020 \( \pm \) 12.016    \\
      SR     & DY+Jets  & 10051.850 \( \pm \) 999.582  & 9597.739 \( \pm \) 594.433  & 9641.194 \( \pm \) 542.019  \\
      SR     & Top      & 926.759 \( \pm \) 71.943     & 967.952 \( \pm \) 63.703    & 973.730 \( \pm \) 59.649    \\
      SR     & VBS\_EWK & 51.194 \( \pm \) 3.726       & 52.461 \( \pm \) 107.649    & 0.000 \( \pm \) 0.000       \\
      SR     & VBS\_QCD & 442.505 \( \pm \) 35.644     & 456.933 \( \pm \) 31.622    & 459.409 \( \pm \) 31.584    \\
      \midrule
      \multicolumn{5}{c}{Resolved ZV 2018}                                                                         \\
      \midrule
      CR     & DY+Jets  & 12231.940 \( \pm \) 654.622  & 11680.437 \( \pm \) 153.086 & 11701.037 \( \pm \) 155.683 \\
      CR     & Top      & 1134.261 \( \pm \) 60.444    & 1264.976 \( \pm \) 68.334   & 1273.043 \( \pm \) 66.040   \\
      CR     & VBS\_EWK & 27.689 \( \pm \) 1.476       & 30.878 \( \pm \) 48.320     & 0.000 \( \pm \) 0.000       \\
      CR     & VBS\_QCD & 345.569 \( \pm \) 18.418     & 385.386 \( \pm \) 20.806    & 387.844 \( \pm \) 20.136    \\
      SR     & DY+Jets  & 13006.154 \( \pm \) 1468.349 & 12740.173 \( \pm \) 609.156 & 12810.317 \( \pm \) 509.178 \\
      SR     & Top      & 1393.580 \( \pm \) 126.941   & 1564.524 \( \pm \) 108.896  & 1579.267 \( \pm \) 95.428   \\
      SR     & VBS\_EWK & 87.014 \( \pm \) 6.599       & 97.074 \( \pm \) 149.836    & 0.000 \( \pm \) 0.000       \\
      SR     & VBS\_QCD & 712.871 \( \pm \) 66.603     & 817.870 \( \pm \) 59.088    & 827.619 \( \pm \) 51.458    \\
      \bottomrule
    \end{tabular}}\label{tab:fit-values-zjj}
\end{table}