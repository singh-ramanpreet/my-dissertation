\chapter{
  Event Simulation and Reconstruction
 }\label{ch_reco}

The proton-proton collision at \gls{LHC} produces shower of particles, before
the event information can be easily used in an analysis, the data collected goes
through iterative process of reconstruct particles produced in collision.
\gls{CMS} uses \gls{PF} algorithm to reconstruct 4-vectors of muons, electrons,
photons, hadrons, jets and missing transverse momentum~\cite{cms-particle-flow-2017}.

To analyze the data collected and compare it with theoretical model, events are
simulated using \gls{MC} event generators and are passed through detector simulation
and \gls{PF} so that \gls{MC} events can be treated same as real events.

This chapter describes the basic ingredients for object reconstruction, \gls{PF}
candidates and \gls{MC} event generators used in this analysis.

\section{
  Track Reconstruction and Calorimeter Clustering
 }\label{ch_reco:track-calo}

For complete particle reconstruction two main ingredients are tracks left by particle
in the detector, and energy deposit in calorimeter. This section describes track reconstruction
from the hits in Tracker and Muon Detector, and energy deposit measurement from
calorimeter clustering.

Track reconstruction requires reconstructed hits, and seed generation which are
described in~\cite{cms-track-vertex}, then the track reconstruction is done using
pattern recognition which is based on combinatorial \gls{KF} method~\cite{cms-track-reco}.
It is an iterative process starting from seed layer the track is estimated and then
proceeds to next layer one by one, at each successive layer the track trajectory
is better known. There can be multiple hits in each new layer, for this multiple
trajectory candidates are created. All the trajectory candidates are grown in parallel
to avoid bias, and truncated at each layer to prevent exponential increase
in number of candidates. Then finally the track is fitted to compute momentum
and vertex information.

The main purpose of calorimeter clustering is to determine position and
energy deposit of the particle. A cluster in a calorimeter is a local group of
energy deposits that are spatially consistent with a electromagnetic
or hadronic shower. First the topological
clusters are identified, a topological cluster is a contiguous region of energy
deposit, then a seed is identified in topological cluster with certain energy
threshold, and highest among the 8 neighbors for \gls{ECAL} and 4 neighbors for \gls{HCAL}.
Now starting with seed energy and position, the neighbors energy are added, and
new position is calculated. For the case when we have just one seed, there is only
iteration until all the neighbors are added, in case when we have more than one seeds
in a cluster, the energy from neighbors is shared, the fraction of energy shared
depends on the energy and position of the cluster, after the first
iteration of calculation of the energy and position the
process is repeated with new values of cluster's energy and
position until either the maximum iteration is reached or cluster's energy and position
values are converged.

\section{
  Reconstructed Particles
 }

After tracks and calorimeter clusters are formed, \gls{PF} links this information
from the detectors together to form objects as broadly discussed in
Section~\ref{ch_cms:cms} and shown in Figure~\ref{fig:cms-slice}. This section
describes the properties of those reconstructed particle candidates.

\subsection{
  Muons
}

Reconstructing muon with best precision is the key ingredient for many physics
searches. Muons reconstruction and identification uses all the information from tracker,
calorimeters and muon detector. There are two types of reconstruction performed
``Global'' and ``Tracker'' for muon candidates. Global muons are formed combining and refitting
muon hits in the muon detector with compatible track from \gls{ST}, and the tracker muons
are formed by extrapolating tracks from \gls{ST} to segments in muon detector.

Once the muon candidates are found, the kinematics properties (\( p_T, \eta, \phi \))
are calculated from track fitting, and other properties such as distance form
\gls{PV} \( dxy \), \( dz \), number of hits in the tracker and muon system, tracker
based relative isolation (\ref{eq:trackerRelIso-muon}) in a cone of \( \Delta R = 0.3 \), and
\gls{PF} relative based isolation (\ref{eq:pfRelIso-muon})
in a cone of \( \Delta R = 0.4 \) are stored for cleaning and isolating muons for
physics analysis.

The tracker and \gls{PF} based relative isolation are defined as,
%
\begin{equation}\label{eq:trackerRelIso-muon}
  \text{TkIso03} = \left( \sum p_{T}^{\text{Tracks (PV)}} \right) /
  \left( p_{T}^{\mu} \right)
\end{equation}
%
\begin{equation}\label{eq:pfRelIso-muon}
  \text{PFRelIso04} = \left( \sum p_{T}^{\text{CH (PV)}}
  + \min \left[ 0, \sum E_{T}^{\text{NH}} + \sum E_{T}^{\gamma}
    - 0.5 \sum p_{T}^{\text{CH (PU)}} \right] \right) /
  \left( p_{T}^{\mu} \right)
\end{equation}

where ``Tracks (PV)'' refers to all the tracks in tracker and coming from \gls{PV},
``CH (PV)'' and ``CH (PU)'' refers to charged hadrons coming from \gls{PV} and \gls{PU}
respectively, ``NH'' refers to neutral hadrons, \( \mu \) refers to muon, and
\( \gamma \) refers to photon.

There are multiple source of muons whenever collision event happens, they can be
real muons or hadrons which are misidentified as muons, these hadrons
are able ``punch'' through \gls{HCAL} and leaves hit in muon detector. The real
muons of interest are called ``prompt'' muons and others are either usually referred
as ``fake'' or ``non-prompt''. Fake muons can originate from decay of pions and kaons in flight
usually identified with a ``kink'' in track or from heavy flavor decay of b or c-quarks
which are identified with tracks not originating form \gls{PV}.
The prompt muons are the ones coming from decay of H, W, Z bosons and \( \tau \) leptons,
and have small impact parameter from \gls{PV}, have hits in both tracker
and muon detector, and are typically well isolated.

In addition to muons from collision events, there can be cosmic muons from pion decay in
upper atmosphere. Cosmic muons are generally not in-time with collision and far from
interaction points.

\subsection{
  Electrons and Photons
}

Since there is large amount of material in tracker, electrons often emit bremsstrahlung
photons when passing through tracker volume,
and photons can further decay to \( e^- e^+ \) pair which complicates the
tracking algorithm. The energy deposit of such electrons
emitting bremsstrahlung will have large spread in \( \phi \) direction because
the magnetic field will bend electrons in \( \phi \) whereas photons are unaffected.
For this reason electron and photon reconstruction are done together, and
the \gls{GSF} algorithm is used for electron track reconstruction
which takes care of kinks in electrons track because of hard emission~\cite{cms-electron-gsf}.

An electron is reconstructed when an \gls{ECAL} cluster matches a \gls{GSF} track,
and a photon is reconstructed when an \gls{ECAL} cluster with \( E_T \)
more than 10\GeV{} is found and have no matching \gls{GSF} track. To prevent
electron and photon from being misidentified as jets certain conditions are applied,
for electron the number of \gls{GSF} track matching with \gls{ECAL} cluster is limited to
maximum of two, and energy deposit in a cone of \( \Delta R = 0.15 \) in \gls{HCAL}
around the position of electrons and photons is required to be less than 10\%.

Similar to muons, after electron and photon reconstruction is done, their kinematics properties
are calculated and various other properties required for cut based
and \gls{MVA} based identification are stored. The detailed description of electrons
and photons identification technique and properties used in this dissertation
can be found in Reference~\cite{cms-egamma-id}.

\subsection{
  Hadrons and Jets
}

Quarks and gluons produced in a collision event are not detected directly,
because of color confinement, they go through fragmentation and
hadronization making a collimated spray of particles mostly made of hadrons
and are called ``jets''. Charged hadrons are reconstructed when a \gls{HCAL} cluster can be associated with
one or more tracks, if the track association fails the cluster is reconstructed as
neutral hadron.

Jet in \gls{CMS} are reconstructed using \FASTJET{} package~\cite{fastjet-manual},
which takes input of all \gls{PF} candidates and associated tracks. The clustering
basically combines 4-vectors of particles iteratively and stop when distance between
two particles (\( d_{ij} \)) is higher than stopping distance (\( d_{iB} \)).

\( d_{ij} \) and \( d_{iB} \) are defined as,
%
\begin{equation}
  d_{ij} = \min (p_{Ti}^{2p}, p_{Tj}^{2p}) \frac{\Delta R_{ij}^{2}}{R^{2}}
\end{equation}
%
\begin{equation}
  d_{iB} = p_{Ti}^{2p}
\end{equation}

where \( p \) is the parameter for different clustering algorithms, \( R \) is the
cone size, and
\( \Delta R_{ij} \) is distance between two particles in iteration.

Anti-\( k_T \) (AK) is the most use jet algorithm in physics analysis, this corresponds to
\( p = -1\), this means the hard particles will be clustered first in this clustering
algorithm.
The cone size used for standard jets (AK4) is \( R = 0.4 \), and for large jets (AK8) often called
``fatjet'' is \( R = 0.8 \).

To mitigate the effect of \gls{PU} contamination in jets two most commonly used
techniques are \gls{CHS} and \gls{PUPPI}~\cite{puppi2014}.
\gls{CHS} as the name suggests removes all the \gls{PF} in the jet clustering which are
originating from \gls{PU} vertices, and it is a standard technique for AK4 jets.
\gls{PUPPI} works by identifying \gls{PU} in an event from charged \gls{PU} information,
then it assigns a weight to all the other particles inside jet, such as neutral particles,
the weight is then used to rescale momentum of those particles.
The main limitation of \gls{CHS} is that it only removes charged \gls{PU} contribution,
for larger jets it can be issue, since it is clustering larger number of particles
and can have significant contribution from neutral hadrons, for this reason \gls{PUPPI}
technique is used for AK8 jets.

To improve the jet selection and reject jets originating purely from \gls{PU}
two methods are used in this dissertation, jet identification based on
multiplicities and energy fraction of particles contained in the jet,
and \gls{MVA} based \gls{PU} identification which uses jets shape
variables to discriminate prompt jet from pileup jets.
The details of \gls{PU} mitigation and identification used in
\gls{CMS} are described in Reference~\cite{cms-jme-pu-run2}.

After jets reconstruction is complete the in addition to calculating kinematics
properties (\( p_T \), \( \eta \), \( \phi \) and mass), various other
properties such as b-quark tagging and quark-gluon likelihood are also
calculated and stored.

\subsubsection{
  N-Subjetiness and Deep Taggers
}

The origin of fatjets are usually when heavy energetic particle
often referred to as ``boosted'' decays hadronically,
for example boosted W or Z bosons decaying to a pair of quarks.
To find and discriminate the fatjet of interest based on in it's substructure
the two technique studied and used are N-Subjetiness~\cite{tau21-paper}
and ``deep tagger''~\cite{cms-jme-deep-tagger}.

N-Subjetiness is defined as,
%
\begin{equation}
  \tau_N = \frac{1}{d_0} \sum_k p_{T,k} \min \{ \Delta R_{1,k}, \Delta R_{2,k}, \ldots , \Delta R_{N,k} \}
\end{equation}

where \( k \) runs over constituent particles in a jet, \( \Delta R_{J,k} \)
is the distance between subjet \( J \) and \( k \) constituent, and \( d_0 \)
is the normalization constant defined as,
%
\begin{equation}
  d_0 = \sum_k p_{T,k} R_0
\end{equation}

\( \tau_N \) quantifies to what degree a jet can be regarded as made of \( N \) jets.
The small values of the \( \tau_N \) means a jet is more likely to have
\( N \) or less subjets, and higher value means it will at least have \( N + 1\)
subjets. Rather than using \( \tau_N \) alone, ratio of different \( \tau_N \)
variables is used, which more discriminating for cases like W vs QCD jets.
Figure~\ref{fig:cms-tau21-tau32-comparison} shows distribution of
\( \tau_{21} \) and \( \tau_{32} \) shapes in signal and background.
\( \tau_{21} \) is used to discriminate fatjets with 2-prongs (W/Z/H)
and \( \tau_{32} \) with 3-prongs (t-quark) substructure against QCD jets.

\begin{figure}[!ht]
  \centering
  \includegraphics[width=\textwidth]{figures/CMS_JME_18_002_Figure_003.pdf}
  \caption[Comparison of \( \tau_{21} \) and \( \tau_{32} \) shapes for signal and background in AK8 jets]%
  {Comparison of \( \tau_{21} \) and \( \tau_{32} \) shapes for signal and background in AK8 jets.
    The left is \( \tau_{21} \) distribution showing discrimination W/Z/H jets
    vs QCD jet, and the right is \( \tau_{32} \) distribution for t-quark vs
    QCD jets~\cite{cms-jme-deep-tagger}.}%
  \label{fig:cms-tau21-tau32-comparison}
\end{figure}

Deep Tagger for AK8 are \gls{ML} based tagger developed to determine
origin of a fatjet. These taggers are trained on particle level information
from \gls{PF} and provide multi class tagging probabilities. In addition to
there are version of these taggers is also which is de-correlated from the mass of jet,
this is important for analysis including this dissertation where we utilize
mass regions of fatjet to normalize background contribution. Figure~\ref{fig:cms-deepAK8-arch}
describes the architecture of ``DeepAK8'' tagging.

\begin{figure}[!ht]
  \centering
  \includegraphics[width=0.9\textwidth]{figures/CMS_JME_18_002_Figure_009.pdf}
  \caption[The network architecture of DeepAK8]%
  {The network architecture of DeepAK8~\cite{cms-jme-deep-tagger}}%
  \label{fig:cms-deepAK8-arch}
\end{figure}

\subsubsection{
  Softdrop Mass
}

Fatjets can also have contamination coming from wide angle
soft \gls{ISR} and multiple hadron scattering,
which affects the mass calculation of the jet, to remove
such contamination and have better mass reconstruction, the
``softdrop'' mass algorithm~\cite{softdrop-mass-2014} is used.

Softdrop is a declusttering algorithm which removes the particle from the jet
with radius \( R_0 \), when the following condition between two particles is satisfied,
%
\begin{equation}
  \frac{\min(p_{T,1}, p_{T,2})}{p_{T,1} + p_{T,2}} > z_{cut} {\left( \frac{\Delta R_{12}}{R_0} \right)}^{\beta}
\end{equation}

where \( \Delta R_{12} \) is the distance between the two particles, \( z_{cut} \)
and \( \beta \) are the parameters for tuning softdrop declusttering.
For fatjets used in \gls{CMS}, they have softdrop applied with
\( \beta = 0\) and \( z_{cut} = 0.1\) which vetoes both soft
and soft-collinear emissions in a jet.

\subsection{
  Missing transverse momentum
}

Invisible particles like neutrinos cannot be detected at \gls{CMS} directly.
Kinematics of such particles can be determined using laws of conservation of
total momentum. In case of proton-proton collision,
the actual collision happens between quarks contained in
proton and quarks carry fraction of proton momentum
and can not be determined exactly, for this reasons kinematic determination
of invisible particles is limited to transverse plane only.

After all the particles are reconstructed in an event, their \( p_T \)'s
can be used to determine missing transverse momentum as,
%
\begin{equation}
  \vec{p}_{T}^{~miss} = - \sum \vec{p}_{T}
\end{equation}

It's usually neutrinos which contributes to missing transverse momentum,
and they have very small mass, the missing transverse momentum is then
equivalent to \gls{MET}, which is most often used term in physics analysis.

\section{
  Monte Carlo Simulation
 }

\subsection{
  Generators
}

\subsection{
  Hadronization
}
