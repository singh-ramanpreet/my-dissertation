\chapter{
  Event Simulation and Reconstruction
 }\label{ch_reco}

The proton-proton collision at \gls{LHC} produces shower of particles, before
the event information can be easily used in an analysis, the data collected goes
through iterative process of reconstruct particles produced in collision.
\gls{CMS} uses \gls{PF} algorithm to reconstruct 4-vectors of muons, electrons,
photons, hadrons, jets and missing transverse momentum~\cite{cms-particle-flow-2017}.

To analyze the data collected and compare it with theoretical model, events are
simulated using \gls{MC} event generators and are passed through detector simulation
and \gls{PF} so that \gls{MC} events can be treated same as real events.

This chapter describes the basic ingredients for object reconstruction, \gls{PF}
candidates and \gls{MC} event generators used in this analysis.

\section{
  Track Reconstruction
 }

\gls{KF}~\cite{cms-track-reco} \GEANTfour{}

\section{
  Calorimeter Clustering
 }

\section{
  Particles Flow Candidates
 }

\subsection{
  Muons
}

\subsection{
  Electrons and Photons
}

\subsection{
  Hadrons and Jets
}

\subsection{
  Missing transverse momentum
}

\section{
  Monte Carlo Simulation
 }

\subsection{
  Generators
}

\subsection{
  Hadronization
}
