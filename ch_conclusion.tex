\chapter{
  Conclusions
 }\label{ch_results}

In this dissertation the contribution towards
development of \gls{SM} \gls{VBS}
analysis in semileptonic \textit{ZVjj} channel are discussed, and work
done towards development and instrumentation of new the detector
\gls{HGCAL} for the \gls{CMS} experiment are also discussed.

This analysis is done with the 137 \fbinv{} of integrated luminosity data
using 13\TeV{} proton-proton collision
dataset collected by the \gls{CMS} experiment during the run period 2016 to 2018.
\gls{MVA} approach was used to
model signal versus background classifier using gradient boosted
\gls{BDT} in the signal region. To correct
and normalize DY plus jets background model, a control region
defined using hadronic boson mass was used.
Expected significance of \( ~1.5\sigma \) is reported for
\gls{EW} \gls{VBS} \textit{ZVjj}.
The analysis is currently being developed further
and under consideration for pre-approval by the Standard Model Physics
group of the \gls{CMS} collaboration.

For analysis like this which requires higher jet multiplicity
in an event, the sensitivity of the analysis suffers
greatly from the pileup contamination.
\gls{HGCAL} in addition to replacing dying \gls{ECAL} and \gls{HCAL}
hardware, it will also help many \gls{VBS} analysis,
since the jets in endcap suffers the most from pileup
contamination, by allowing to construct narrow jets
using lateral and longitudinal granularity of the silicon and scintillator cells,
pileup contamination can be significantly reduced.
\gls{HGCAL} is expected to installed
during \gls{LS3} which is currently expected to be from end of year 2025
to the start of year 2029.

In Chapter~\ref{ch_hgcal} of this dissertation, optimal configuration
of scintillator tiles coupled with \glspl{SiPM}
were studied and suggested by simulating end-of-life scenarios
of the \gls{HGCAL}. Results made use of testbeam measurement of
the scintillator tiles conducted by \gls{FNAL}
and the cold noise measurement of \glspl{SiPM}.

To wrap scintillator tiles with
\gls{ESR} film is challenging task because of some inflexibility
in it. To wrap more than hundred thousand
for already difficult task requires
automation to wrap faster with repeatability.
Complete automated wrapping of scintillator tiles with
\gls{ESR} with wrapping machine at \gls{NICADD} were also discussed
in the Chapter~\ref{ch_hgcal}.