\chapter{
  Summary
 }\label{ch_results}

In this dissertation the related contribution of the
development of \gls{SM} \gls{VBS}
analysis in semileptonic \textit{ZVjj} channel, and
the development and instrumentation of the upgraded endcap calorimeter
(\gls{HGCAL}) for the \gls{CMS} experiment are discussed.
Measurement of \gls{VBS} is key to understanding the nature of \gls{EWSB} at the \TeV{} scale.
Higher integrated luminosity and the \gls{HGCAL} upgrade will improve the signal sensitivities of VBS like searches
in the future.

This \gls{SM} \gls{VBS} analysis is done with 137 \fbinv{} of
using 13\TeV{} proton-proton collision
data collected by the \gls{CMS} experiment during the 2016 to 2018 run period.
The \textit{Z} boson decays to a pair of leptons (electrons or muons),
and the other vector boson \textit{V} (either \textit{W} or \textit{Z})
decays hadronically to a pair of jets, in association of two large pseudorapidity jets.
In this channel DY plus jets is the dominant background.
To correct
and normalize DY plus jets background model, a control region
defined using hadronic boson mass was used.
\gls{MVA} approach was used to
model the signal versus background classifier using gradient boosted
\gls{BDT} in the signal region.
Expected significance of \( ~1.5\sigma \) is reported for
\gls{EW} \gls{VBS} \textit{ZVjj}.

The \gls{HGCAL} in addition to replacing the dying \gls{ECAL} and \gls{HCAL}
hardware, will also enhance many \gls{VBS} analyses.
Since the jets in endcap suffer the most from pileup
contamination. Reconstructing narrow jets
using the lateral and longitudinal granularity of the silicon and scintillator cells,
can significantly increase the physis sensitivity of the \gls{CMS} data.
\gls{HGCAL} is expected to be installed
during \gls{LS3} which is currently scheduled to be from end of year 2025
to the start of year 2029. Simulation and instrumentation studies,
critical to the configuration and assembly of the scintillator part of
the \gls{HGCAL} are reported on.